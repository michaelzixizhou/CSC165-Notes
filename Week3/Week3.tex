\documentclass{article}
\usepackage[top=2.54cm, bottom=2.54cm, outer=8cm, inner=2.54cm, heightrounded, marginparwidth=4.46cm, marginparsep=1cm]{geometry}
\usepackage{lipsum}
\usepackage{parskip}
\usepackage{amsfonts}
\usepackage{marginfix} % floats sidenotes
\usepackage{sidenotes}
\usepackage{amsmath}
\usepackage{amssymb}
\usepackage [english]{babel}
\usepackage [autostyle, english = american]{csquotes}
\MakeOuterQuote{"}
\usepackage{enumitem}
\title{Week 3: Induction and Representations of Natural Numbers}
\author{Michael Zhou}
\date{2023-01-07}

\newcounter{defcount}
\newcounter{excount}
\newcounter{thcount}
\newcommand\df{\stepcounter{defcount} \textbf{Definition 3.\thedefcount}. }
\newcommand\ex{\stepcounter{excount} \textbf{Example 3.\theexcount}. }
\newcommand\tr{\stepcounter{thcount} \textbf{Theorem 3.\thethcount}. }
\newcommand\qedsymbol{\hfill$\blacksquare$}


\begin{document}
\maketitle
Use induction to prove:
\begin{itemize}
    \item Statements in number theory 
    \item Statements in new domains 
    \item Properties about sequences 
    \item Counting combinatorial objects
\end{itemize}

\section{The principle of induction}
\df The principle of \textbf{induction} can be used to prove statements 
of the form $\forall n \in \mathbb{N}, P(n)$, and \textit{only} applies to 
statements of this type --- universal quantifiers over natural numbers. 

The most bjasic form, \textbf{simple} induction, has two steps:\sidenote{
These steps are called \textbf{conjunctions}, and may be presented in either 
order.}
\begin{itemize}
    \item The \textbf{base case} is the proof that the statement holds for the 
        first natural number, 0. 
    \item The \textbf{inductive step} is a proof that for all $n \in \mathbb{N}$, 
        if $P(n)$ is True, then $P(n+1)$ is also True: 
        $$\forall n \in \mathbb{N}, P(n) \Rightarrow P(n+1)$$
\end{itemize}
Once these two steps are proven, then by the principle of induction, we can 
conlcude that $\forall n \in \mathbb{N}, P(n)$.

\ex Prove that for any $m, x, y, n \in \mathbb{N}$ such that $n \geq 1$ if 
$x \equiv y \text{ mod } m$, then $x^n \equiv y^n \text{ mod } m$.

This is a question from PS1, and we could prove that $x^2 \equiv y^2 \text{ mod } m$ 
and $x^3 \equiv y^3 \text{ mod } m$ and so on by the same pattern of 
$ac \equiv bd \text{ mod } n$. But to make it mathematically rigourous for 
all $n$ and $m$, we need to use induction.

How does this work? It is essentially the domino effect that we've examined 
in PS1. By knowing $P(0)$ is True, we know that $P(0) \Rightarrow P(1)$ through 
the base case, then we know $P(1) \Rightarrow P(2)$ by the inductive step, 
and so on. 

By showing that as long as it is True for $n$, it is also True for the 
number right after $n$, then we can conclude that it is True for every $n$.

\newpage
\section{Examples from number theory} 
\ex Prove that for every natural number $n$, $7 \mid 8^n - 1$.

\textit{Translation.} $\forall n \in \mathbb{N}, 7 \mid 8^n - 1$\sidenote{
    The translation and discussion section will start merging into one soon.
}

\textit{Discussion.} Definite the predicate $P(n)$ as "$7 \mid 8^n - 1$", 
then by the divisibility predicate, $7 \cdot y = 8^n - 1$. Then we know 
exactly how to use induction in this proof.

\textit{Proof.} Let $P(n)$ be $7 \cdot y = 8^n - 1.$
$$\exists y \in \mathbb{Z}, 7 \cdot y = 8^n - 1$$
We want to prove that for all $n \in \mathbb{N}$ that $P(n)$ holds.

\textbf{Base case:} Let $n = 0$, we want to prove $P(0)$ holds True. \\
We know that $7 \cdot 0 = 8^0 - 1$ holds True.

\textbf{Inductive step:} Let $k \in \mathbb{N}$, and assume that $P(n)$ 
is True, we want to prove $P(n+1)$. In other words, 
$$7 \mid 8^{k+1} - 1 \text{ or } \exists y \in \mathbb{Z}, 8^{k+1} - 1 = 
7y_k$$
How do we prove this? Well we want to use previously proved definitions, 
so extracting $8^k - 1 $ from the equation could work.
\begin{align*}
    8^{k+1} - 1 &= 8^{k+1} - 8 + 7 \\
                &= 8(8^k - 1) + 7 
\end{align*}
Next we can use the \textbf{induction hypothesis}, which is $7y_k = 8^k - 1$.
\begin{align*}
    8^{k+1} - 1 &= 8(8^k - 1) + 7 \\
                &= 8(7y_k) + 7 \\
                &= 56y_k + 7 \\
                &= 7(8y_k + 1)  
\end{align*}
We can let $8y_k + 1 = y_{k+1}$, and thus the proof is complete.

\ex Prove that for every natural number $n, n(n^2 + 5)$ is divisible by 6.

\textit{Proof.} Let $P(n)$ be the statement that $\exists n \in \mathbb{N}, 
6 \mid n (n^2 + 5)$.

\textbf{Base case:} Prove $P (0)$ is True. We know this is True since
$6 \mid 0 (0^2 + 5) = 6 \mid 0$, and anything divides 0.  

\textbf{Inductive step:} Let $k \in \mathbb{N}$, and assume $P(k)$ is True. 
We want to prove $P (k+1)$ holds, such that $ 6 \mid (k + 1) ((k+1)^2 + 5)$. 
We are looking to isolate a $k(k^2 +5)$ term to apply the induction 
hypothesis.
\begin{align*}
    (k+1) ( (k+1)^2 + 5) &= (k+1) (k^2 + 2k + 6) \\
                         &= (k+1) ((k^2 + 5) + (2k + 1)) \\
                        &= k(k^2 + 5) + k(2k + 1) + (k^2 + 5) + (2k + 1)\\
                        &= k(k^2 + 5) + 3k^2 + 3k  + 6 \\
                        &= k(k^2 + 5) + 3k(k+ 1) + 6 
\end{align*}

For the first term $k(k^2 +5)$, we know that is divisible by 6 due to the 
induction hypothesis. 

The second term is tricky, but we know that $k(k + 1)$
has two consecutive natural numbers, meaning one of them is even, so the 
entire term is even. What does this imply? It means that $ k(k+1)$ has a
factor of 2. Adding in the coefficient of 3, it means that the second term 
has a factor of 6 for certain. 

The third term is obviously divisible by 6. 

Using the divisibility of linear combinations, since each term is divisible 
by 6, the entire sum is also divisible by 6. Thus, the proof is complete.

Finally let's go back to Example 3.1 to prove it using induction.

\ex Prove that for any $m, x, y \in \mathbb{N}$ and for any $n \in \mathbb{N}$ 
that if $x \equiv y (\text{ mod } m)$ we have $x^n \equiv y^n (\text{ mod } m)$.

\textit{Translation.}\sidenote{Note that we separated the variables $m, x, y$ 
and $n$, we will discuss this in the next section.}
$$\forall m, x, y \in \mathbb{N}, \forall n \in \mathbb{N},  x \equiv y 
(\text{mod } m) \Rightarrow \forall n \in \mathbb{N}, x^n \equiv y^n 
(\text{ mod } m)$$
\textbf{Base case:} Prove $P (0)$ is True. We have $x^0 \equiv y^0 
(\text{ mod } m)$, which is trivially True.

\textbf{Inductive step:} Let $k \in \mathbb{N}$. Assuming $P(n)$ is True, 
prove $P(n+1)$ is True. We have $x^{k+1} \equiv y^{k+1} (\text{ mod } m)$. 

From the previous example, we know that
$$x^k \equiv y^k (\text{ mod } m) \land x \equiv y (\text{ mod } m) 
\Rightarrow x \cdot x^k \equiv y \cdot y^k (\text{ mod } m)$$

Since we assumed the left side is True, we also know the right side 
is True due to the implication. Thus, the proof is complete.

Why did we exclude the $n$ from the first quantification? 

Changing the order of quantifiers can change the proof that we could write: 
\begin{itemize}
    \item Another way we could've written this is 
        $$\forall n \in \mathbb{N}, \forall m, x, y \in \mathbb{N}, 
        x \equiv y (\text{ mod } m) \Rightarrow x^n \equiv y^n (\text{ mod } m)$$
        Then we can define $P(n)$ to exclude the $\forall n \in \mathbb{N}$ 
        portion. 
    \item By convention, it would not make sense if we partitioned the 
        the $\forall n \in \mathbb{N}$ and the conditions to define it 
        as $P (n)$, so it would be best if we wrote it the same way as 
        we planned to define it 
\end{itemize}

\ex Prove that for all natural numbers $n$ greater than or equal to 3, 
$2n + 1 \leq 2^n$.

\textit{Translation.} $\forall n \in \mathbb{N}, n \geq 3 \Rightarrow 2n + 1 
\leq 2^n$. We can assume the hypothesis is True and prove the conclusion.

\textit{Proof.} Let $n \in \mathbb{N}$, and let $2n + 1 \leq 2^n$ 
be $P(n)$. We can assume that $P(n)$ is True. WTS $P(0)$ is True, 
and $P(n+1)$ is True through induction.

\textbf{Base case.} $P(0)$ is $2^0 + 1 \leq 2^0$, which is $2 \leq 1$, 
thus it is True.

\textbf{Inductive step.} Let $k \in \mathbb{N}$, and $k = n$. We assume 
$P(k): 2k + 1 \leq 2^k$. WTS $P(k+1)$ is True: $2(k+1) + 1 \leq 2^{k+1}$. 
\begin{align*}
    2k + 1 &\leq 2^{k} \\
    2k + 1 + 2 &\leq 2^k + 2\\
    2(k+1) + 1 &\leq 2^{k+1}
\end{align*}

\ex Prove that for all $n \in \mathbb{N}$, if $n \geq 1$. then $2^{2^n}-1$ 
is divisible by at least $n$ distinct primes.

\textit{Proof.} Let $k \in \mathbb{N}$, then $P(k)$ is 
$n \geq 1 \implies (Prime(k) \mid 2^{2^k} - 1)$. 

\textbf{Base case.} $P(1)$ is $2^{2^1} - 1 = 3$, which is divisible by 3, 
thus it is True.

\textbf{Inductive step.} We assume $P(k)$ to be True. WTS $P(k+1)$ is True: 
$2^{2^{k+1}} - 1$.
\begin{align*}
    2^{2^{k+1}} - 1 &= 2 ^{2^{k} \cdot 2 } - 1 \\
                    &= (2^{2^k} - 1) (2^{2^k} + 1 ) 
\end{align*}
We know the left term is True since that is $P(k)$, and the right term 
is an odd number. The difference between the two numbers is 2, so the gcd 
must be either 1 or 2, but it cannot be 2 since the right term is odd. Thus 
the right term introduces a new prime number since the gcd is 1, and proves 
$P(k+1)$.

\section{Combinatorics}
\df \textbf{Combinatorics} is an area concerned with counting objects and
analyzing patterns. A \textbf{pattern} is a sequence of numbers that we 
often want to derive a closed-form expression for $a_k$ (the $k^{th}$ 
number in the sequence) or for $\sum_{i=0}^{k} a_i$ (the sum of the first 
$k+1$ numbers in the sequence).

\ex Consider the following sequence:
$$0, 1, 1, 2 ,3 , 5, 8, \ldots $$
Does it look familiar? The \textit{Fibonacci sequence} is a famous pattern of 
combinatronics: for all $k \geq 2$, $a_k = a_{k-1} + a_{k-2}$.

\ex Another example is the \textit{arithmetic sequence}. Suppose we start 
with \$20, and we gained \$200 every month. How much money would we 
have in $k$ months? 
$$a_0 = 20, a_1 = 220, a_2 = 420, \ldots$$
We get the sequence of $a_k = 20 + 200k$.\sidenote{This is also $y = mx + b$.}

\ex Suppose we started with \$10, and our money doubled every month. How 
much would we have in $k$ months? 
$$a_0 = 10, a_1 = 20, a_2 = 40, \ldots$$
We get a \textit{geometric sequence} of $a_k = 10 \cdot 2^k$. 

\newpage 
\ex Suppose we wanted to add all $n \in \mathbb{N}$ up to and including 
$k$, it would give rise to the following:
$$a_k = 0 + 1 + \ldots + k$$
We can express this infinite sequence as a closed form expression 
$a_n = n \cdot \frac{ (n + 1)}{2}$. 

\subsection{Proofs of closed form expressions}
\df In general, a \textbf{sequence} is an ordered list of numbers such that 
$f: \mathbb{N}\rightarrow \mathbb{R} $ where $a(0) = f(0)$ and so on. 

\df A sequence is \textbf{closed form} or \textbf{explicit} when there is a
fixed number of elementary operations. We can often use induction to prove 
that an explicit formula computes the terms in a sequence.

For example, the Fibonacci sequence also has an explicit formula called 
Binet's formula: 
$$a_n = \frac{(1 + \sqrt{5}) ^n - (1 - \sqrt{5} )^n}{2^n \sqrt{5}}$$

\ex Use induction to prove the first $n$ positive integers is equal to 
$n \cdot \frac{ ( n - 1)}{2}$

\textit{Translation.} $$\forall n \in \mathbb{N}, \sum_{j=1}^{n} j= n \cdot \frac{ ( n + 1)}{2}
$$

\textit{Proof.} Let $P(n)$ be $\sum_{j=1}^{n} j= n \cdot \frac{ ( n + 1)}{2}$.

\textbf{Base case.} Let $n  = 0$, then the left is an empty sum, and the 
right is $0 \cdot \frac{0-1}{2} = 0$. Thus the base case is True.

\textbf{Inductive step.} Let $k \in \mathbb{N}$ and assume that $P(k)$ 
is True: $\sum_{j=1}^{k} j= k \cdot \frac{ ( k + 1)}{2}$. WTS that
$P(k+1)$ is True: $\sum_{j=1}^{k+1} j= \frac{(k+1) ( k+ 2)}{2}$.
\begin{align*}
    \sum_{j=1}^{k} j&= k \cdot \frac{ ( k + 1)}{2} \\
    \sum_{j=1}^{k + 1} j &= k \cdot \frac{ ( k+1)}{2} + (k+ 1) \\
                         &= \frac{ k( k+1)}{2} + \frac{2(k+ 1)}{2} \\
                         &= \frac{k (k + 1) + 2(k+1 )}{2} \\
                         &= \frac{ ( k + 1) (k+2)}{2} 
\end{align*} 
Since we know that the sum we are expressing is $a_k = 1 + 2 + \ldots 
k$, to add the $k^{th}$ term is just adding $(k+1)$ to the sum.

\qedsymbol
\newpage
\subsection{Strengthening the hypothesis}
\ex Prove that the sum of the first $n$ odd numbers is a perfect square.

\textit{Translation.} $$\forall n \in \mathbb{N}, \exists x \in \mathbb{N}
, \sum_{i = 0}^{n - 1} (2i + 1) = x^2
$$

\textit{Discussion.} To prove the inductive step, we can assume that $P(k)$ 
is True. But we run into an issue:
\begin{align*}
    \sum_{i=0}^{k} (2i + 1) &= x^2 \\
    \sum_{i=0}^{(k+1)-1} (2i+1) &= x^2 + (2k + 1) 
\end{align*}
How can we prove that the $2k + 1$ term adds into the right term to make it
a perfect square? We can explore examples to learn more: $1^2 = 1$, $2^2 = 
3 + 1$, $3^3 = 5 + 3 + 1$. We can conjecture that the sum of the first $n$ 
odd numbers is equal to $n^2$. So, we can make our hypothesis even stronger: 
$$\forall n \in \mathbb{N}, \sum_{i=1}^{n-1} (2i + 1) = n^2$$
\textit{Proof.} Let $P(n)$ be the predicate $\sum_{i=1}^{n-1} (2i + 1) = n^2$, 
we will prove that for all $n \in \mathbb{N}, P(n)$ by induction on $n$.

\textbf{Base case.} Let $n = 0 $, then the left is an empty sum and the right 
is 0, thus this is True.

\textbf{Inductive step.} Let $k \in \mathbb{N}$, we will assume that $P(k)$ 
is True: \linebreak $\sum_{i=1}^{k-1} (2i + 1) = k^2$. WTS that $P(k+1)$ is True: 
$\sum_{i=1}^{k} (2i + 1) = (k+1)^2$.
\begin{align*}
    \sum_{i=1}^{k-1} (2i + 1) &= k^2 \\ 
    \sum_{i=1}^{(k+1)-1} (2i + 1) &= k^2 + (2k + 1)\\ 
    \sum_{i=1}^{k} (2i + 1) &= k^2 + 2k + 1 \\
                            &= (k+1)^2
\end{align*}
\qedsymbol
\subsection{Going beyond numbers}
\ex Prove that for every finite set $S$, $\mid \mathcal{P}(s)\mid = 2^{\mid S \mid }
$.\sidenote{Recall that $\mathcal{P}$ denotes a power set --- the set of all 
subsets of $S$}. In other words, a power set of $S$ will have $2^{\mid S \mid} $
number of subsets.\sidenote{$\mid P(n) \mid$ is the size of a set.}

\textit{Translation.} We can perform induction on a variable representing 
the size of a set, since the size of a finite set is always a natural number. 
We can then define a predicate for $n \in \mathbb{N}$:
$$P(n) : \text{Every set $S$ of size $n$ satisfies }\mid \mathcal{P}(S)\mid 
= 2^n$$
Then, our original statement would just be $\forall n \in \mathbb{N}, 
P(n)$, which means we can use induction.

\textit{Proof.} \textbf{Base case.} Let $n = 0$. The only subset of an empty 
set is itself, $\mathcal{P} (S) = \{\emptyset\}$, which is size 1, and $2^0 
= 1$. 

\textbf{Inductive step.} Let $k \in \mathbb{N}$, and assume 
that $P(k)$ is True. WTS $P(k+1)$ is True. Let $S$ be a set with size $k+1$, 
and let $s_1, \ldots, s_{k+1}$ be elements of the set, we wanrt to prove that 
the number of subsets in $S$ is $2^{k+1}$. 

We know that subsets of $S$ that do not contain the last element of $s_{k+1}$ 
have the size $2^{k+1}$ according to our assumption. 

Now consider subsets \textit{only} including the last element --- this is also 
$2^{k+1}$ since you simply add $s_{k+1}$ to $2^k$ subsets. 

Thus, $2^k + 2^k = 2^{k+1}$, and concludes the proof. 

\qedsymbol

\ex Prove that for all $n, m \in \mathbb{N}$, and for all sets $A$ and $B$ 
of size $n$ and $m$, respectively, $\mid A \times B \mid = n \cdot m$.

\textit{Translation.} We can follow the above example, and perform induction 
on a variable representing the size of a set, and we only need to perform 
induction on one of them, as they are both natural numbers. We define a 
predicate for $n \in \mathbb{N}$:
$$P(n) : \text{Every set $A, B$ of size $n$, $m$ satisfies $\mid A \times 
B \mid = n \cdot m$}$$
Then our statement is $\forall n,m \in \mathbb{N}, P(n)$. We will prove for 
$n$.

\textit{Proof.} \textbf{Base case.} Let $ n = 0 $, then $A$ must be an 
empty set, and the Cartesian product would be an empty set, which is size 0.
Thus, $0 = 0 \cdot m$.

\textbf{Inductive step.} Let $k \in \mathbb{N}$ and assume that $P(k)$ holds. 
WTS $P(k+1)$ is True. We can define $A \times B$ of size $k$ and $m$ 
respectively as the set $S_k$:
$$S_k = \{ k_0m_0, 
k_0m_1, \ldots, k_0m_{m-1}, \ldots k_{k-1}m_0, \ldots k_{k-1} m_{m-1} \} $$
And let $S_{(k+1)}$ be the set of $ A \times B$ for only $k + 1$ terms and $m$: 
$$S_{(k+1)} = \{k_km_0 , \ldots, k_k m_{m-1} \}$$
We can express $\mid S_k \mid$ as $k \cdot m$ since we assumed $P(k)$ to be 
True, and thus we can express $\mid S_{(k+1)} \mid$ as $m$ since there is only 
one term of $k + 1$.
\begin{align*}
    \mid A \times B \mid &= k \cdot m \\
                         &= \mid S_k \mid \\
                         &= \mid S_k \mid + \mid S_{(k+1)} \mid \\
                         &= k \cdot m + m \\
                         &= m(k + 1) 
\end{align*}
The proof for $m$ is the same, thus $P(k+1)$ is proven for both $k = n$ and 
$k = m$. 

\qedsymbol

\newpage
\section{Strong induction}
\df \textbf{Strong induction} is a technique that assumes the first $n$ terms 
are all True: $P(0), P(1), \ldots, P(n)$. This is in contrast to \textit{simple
induction}, which was the previously taught method of assuming only $P(n)$. 
These are equivalent. Formally, we want to prove that $$\forall n \in \mathbb{N}, 
n \geq k \Rightarrow P(n)$$
where $k$ is some natural number. We first prove the base case $P(k)$, then 
prove that for any fixed but arbitrary $n \geq k $, $P(j)$ for all $j$ where 
$k \leq j \leq n$ implies $P(n+1)$

\ex Prove that for every integer $n \geq 2$, it can be expressed as a product 
of one or more prime numbers,\sidenote{Sound familiar? This is the prime 
factorization theorem.}

\textit{Proof.} Let $P(n)$ be the statement that $n$ can be expressed as a 
product of one or more prime numbers. 

\textbf{Base case.} Let $n = 2$. 2 can be expressed as itself, which is prime, 
thus $P(2)$ is True.

\textbf{Inductive step.} Let $n$ be an integer, $n \geq 2$. And assume that for 
every integer $j$, $2 \leq j \leq n$, that $j$ can be expressed as a product 
of one or more prime numbers. WTS $P(n+1)$. There are two cases: either 
integer $n + 1$ is a prime number, or it is not. If it is a prime number, then 
it is a product of itself, so this case is complete. 

When it is a not a prime number, it is $n + 1 = a \cdot b$, where $a, b $ are 
positive integers that are both not $n+1$ or $1$. Since $ 2 \leq a \leq n$, 
and $2 \leq b \leq n$, both $a$ and $b$ can be written as a product of prime 
numbers by the induction hypothesis, thus the proof is complete.

\qedsymbol

We would not have been able to show this with simple induction, since all 
we would know is that $P(n)$ can be factorized into primes. 

\end{document} 
