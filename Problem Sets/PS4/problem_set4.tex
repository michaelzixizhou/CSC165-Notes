\documentclass{article}
\usepackage[margin=1in]{geometry}
\usepackage{lipsum}
\usepackage{parskip}
\usepackage{amsfonts}
\usepackage{amsmath}
\usepackage{amssymb} 
\usepackage [english]{babel}
\usepackage [autostyle, english = american]{csquotes}
\MakeOuterQuote{"}
\usepackage{enumitem}
\title{CSC165H1: Problem Set 4}
\author{Michael Zhou}
\date{2023-04-03}

\newcounter{qcount}
\newcommand\q{\stepcounter{qcount} \textbf{Question \theqcount}. }
\newcommand\qedsymbol{\hfill$\blacksquare$}              

\begin{document}
\maketitle
\begin{enumerate}
    \item \textbf{Nested loops.}
        \begin{enumerate}[label=(\alph*)]
            \item Let $s_t$ be the value of variable $s$ and $i_t$ be 
                the value of variable $i$ immediately after $t$ iterations 
                of Loop 1 have occured. Determine a general formula for them 
                and find the exact number of iterations of Loop 1 given 
                value of $n$.

                Looking at $i_t$ and $s_t$ over $t$ iterations, we see a trend:
                \begin{center}
                    \begin{tabular}{| c c c  | }
                        \hline
                        $t$ & $i_t$ & $s_t$ \\
                        \hline 
                        0 & 0 & 1 \\
                        1 & 1 & 3 \\
                        2 & 4 & 5 \\ 
                        3 & 9 & 7 \\
                        4 & 16 & 9 \\
                        5 & 25 & 11\\
                        \hline
                    \end{tabular}
                \end{center}
                We can see that $i_t = t^2$ and $s_t = 2t + 1$. 

                Loop 1 terminates when $ i \geq n$, substituting $i$ we get 
                \begin{align*}
                    t^2 &\geq n \\
                    t &\geq \sqrt{n} \\
                    t &\geq \left\lceil \sqrt{n} \right\rceil 
                \end{align*}
                So Loop 1 runs for $\left\lceil \sqrt{n}
                \right\rceil$ iterations before terminating.  
            \newpage
            \item Give a Theta bound on the running time function $RT (n)$ 
                for this algorithm. Show your work.

                Let $n \in \mathbb{N}$. We know that this algorithm
                \textbf{simply depends on the positive integer $n$, so 
                the best-case would be the same as the worst-case, and the 
                upper bound would be equal to the lower bound.}

                Loop 3 breaks when $k \leq 1$, where $k = n$, and $k$ decreases 
                by a floor division of $2$ every $t$ iterations. So we would 
                have $k_t = \left\lfloor \frac{n}{2^t} \right\rfloor$. Which 
                gives us 
                \begin{align*}
                    \left\lfloor \frac{n}{2^t} \right\rfloor &\leq 1 \\
                    n &\leq 2^t \tag*{(Since 
                    $\left\lfloor \frac{n}{2^t}\right\rfloor \leq \frac{n}{2^t}$)}\\
                    \log_2 n  &\leq t  
                \end{align*}
                Then we would have at least $\log_2 n$ iterations, or 
                in other words $\left\lceil \log_2 n \right\rceil $ 
                iterations for Loop 3 given a fixed number of iterations 
                for Loop 2. This cost remains the same for each iteration of 
                Loop 2.

                Loop 2 breaks when $j \geq i$, and $j_m$ given $m$ iterations 
                is $3m$. So by part (a), given $t$ loops of Loop 1, 
                $3m \geq i_t$, and $m \geq \frac{t^2}{3}$. Which means 
                Loop 2 runs for $\left\lceil \frac{t^2}{3} \right\rceil $ iterations. 
                Expressed as a sum, we get 
                $$ \sum_{j = 0}^{\left\lceil \frac{t^2}{3} \right\rceil } \left\lceil 
                \log n \right\rceil = \left\lceil \log n \right\rceil
                \cdot \sum_{j=0}^{\left\lceil \frac{t^2}{3} \right\rceil } 1 
                = \left\lceil \log n \right\rceil \cdot \left(\left\lceil 
                \frac{t^2}{3}\right\rceil + 1\right) $$
                steps for Loop 2 given fixed iterations of Loop 1. 

                Loop 1 runs for $\left\lceil \sqrt n \right\rceil $ 
                iterations from part (a), so we get the sum 
                \begin{align*}
                    \sum_{i=0}^{\left\lceil \sqrt {n} \right\rceil } 
                \left\lceil \log n \right\rceil
                \cdot \left( \left\lceil \frac{t^2}{3} \right\rceil + 1
                \right) &= 
                \left\lceil \log n \right\rceil 
                \cdot \left( \frac{1}{3} \cdot \sum_{i=0}^{\left\lceil \sqrt {n} \right\rceil }
                 t^2  + \sum_{i=0}^{\left\lceil 
                 \sqrt{n}\right\rceil } 1 \right)  \\
                        &= \left\lceil \log n \right\rceil \cdot  \left(
                            \frac{\left\lceil \sqrt{n} \right\rceil (
                            \left\lceil \sqrt{n}  \right\rceil + 1) (2
                        \left\lceil \sqrt{n} \right\rceil + 1)}{18} + 
                    \left\lceil \sqrt{n} \right\rceil + 1 \right) 
            \end{align*}
            We take the fastest-growing term, which is the first term, to represent 
            asymptotic growth. This is because if $g \in \Theta (h) $ 
            and $f \in \mathcal{O}(g)$, then 
            $f + g \in \Theta (h)$. We can also remove constants and 
            constant factors since it is asymptotic. 
            \begin{align*}
                \left\lceil \log n \right\rceil \cdot \frac{\left\lceil 
            \sqrt{n}\right\rceil (\left\lceil \sqrt{n} \right\rceil + 1) 
        (2 \left\lceil \sqrt{n} \right\rceil + 1)}{18} &= \log n \cdot 
        \sqrt{n} \cdot \sqrt{n} \cdot \sqrt{n} \\ 
                                                       &= \log n \cdot n 
                                                       \cdot \sqrt{n}
            \end{align*}
            Since we know that the upper bound and lower bound are equal, 
            the tight bound on function $RT (n)$ for 
            this algorithm is $\Theta (n \log(n) \sqrt{n})$. 
        \end{enumerate}
    \newpage
    \item \textbf{Worst-case analysis.}
        \begin{enumerate}[label=(\alph*)]
            \item Find, with proof, an upper bound on the worst-case 
                runtime of the algorithm matching the lower bound.

                Given iterations $t$ of the outer while loop and 
                length of the input list $n$, we can 
                see $i$ changes by either $i_t = 2t$ or $i_t = t$, and breaks 
                when $i_t \geq n$. So the outer loop can run for at most 
                $n$ iterations ($i_t$ = t) and at least $\left\lceil 
                \frac{n}{2}\right\rceil $ iterations.
                
                
                The inner loop in the case that $i_t = 2t$ essentially
                resets the loop variable to $i_t$ before running 
                each time, it breaks when $\left\lfloor
                \frac{i_t}{2^t} \right\rfloor$. So in terms of $i_t$, 
                this loop would run for $\left\lceil \log i_t \right\rceil$ 
                times. Writing this as a cost of the outer loop in the worst 
                case we get 
                \begin{align*}
                    \sum_{t=1}^{n} \log 2t &= n\cdot \log 2+ 
                    (\log 1 + \log 2 + \ldots + \log n 
                        ) \\
                            &= n \cdot 
                            \log2 + \log (1 \cdot 2 \cdot 3 \cdot \ldots 
                            n ) \tag*{(Since $\log a + 
                        \log b = log (ab)$)} \\
                            &= n \cdot 
                            \log2 + \log (n!)
                \end{align*}
                So counting the variable initialization steps outside the loop, 
                we have $1 + n \cdot \log2 + \log (n!)$. Taking the 
                fastest growting term we have 
                $$\log (n!) = 1 \cdot 2 \cdot \ldots \cdot n$$
                An upper bound on $\log (n!)$ would be assuming all terms 
                were $n$, which is $\log (n^n) = n \cdot \log n$. We will 
                now prove that this is True.

                Let $c = 1$, $n_0 = 1$. Assume $n \geq n_0$. We will show that 
                $\log (n!) \in \mathcal{O} (n \cdot \log n)$.
                \begin{align*}
                    \log (n!) &= \log n + \ldots + \log 1 \\
                              &\leq \log n + \ldots + \log n \\
                              &= \log (n^n) \tag*{(Since $\log a + \log b 
                              = \log (ab)$)}\\
                              &= 1\cdot n \cdot \log n \\
                              &= c \cdot n \cdot \log n \\
                              &\in \mathcal{O} (n \cdot \log n)
                \end{align*}

                Thus 
                $\log (n!) \in \mathcal{O} (n \cdot \log n )$.

                So the upper bound on the worst-case runtime would be 
                $\mathcal{O} (n \cdot \log n)$.
                
                \qedsymbol
                \newpage
            \item Find, with proof, a lower bound on the worst-case 
                runtime of the algorithm matching the upper bound. 

                Let $n \in \mathbb{N}$. Consider a list $A$ of length $n$ 
                containing only the integer $3$, such that it always enters 
                into the first condition of the algorithm. We want to show 
                that the lower bound on the worst case runtime of 
                this algorithm is $\Omega(n \cdot \log n)$.

                Since all items in $A$ are divisible by $3$, it would 
                always enter the first condition in the while loop. 

                From part (a), we know this will run for at least 
                $\left\lceil \frac{n}{2} \right\rceil $ iterations, 
                with a cost of $\log 2t$ for each $t$ iterations. 
                \begin{align*}
                    \sum_{t=1}^{\left\lceil \frac{n}{2} \right\rceil } 
                    \log 2t &= \left\lceil \frac{n}{2} \right\rceil  \cdot 
                    \log 2 +
                    \log (1 \cdot 2 \cdot 3 \cdot \ldots \cdots \left\lceil 
                    \frac{n}{2}\right\rceil ) \\
                            &= \left\lceil \frac{n}{2} \right\rceil \cdot 
                            \log 2 + \log (\left\lceil \frac{n}{2} \right\rceil !)
                \end{align*}
                Taking the fastest growing term of $\log (\left\lceil 
                \frac{n}{2}\right\rceil! )$, we will prove that this 
                term is $\Omega (n \cdot \log n)$. 

                Let $c = \frac{1}{8}$, $n_0 = 16$. Let $n \in \mathbb{N}$, 
                and assume $n \geq n_0$. We will show that $\log (\left\lceil 
                \frac{n}{2}\right\rceil !) \in \Omega (n \log n)$.
                \begin{align*}
                    \log (\left\lceil \frac{n}{2} \right\rceil! ) &= 
                    \log( \left\lceil \frac{n}{2} \right\rceil) + \ldots + \log 1 \\
                        &\geq \log ( \left\lceil \frac{n}{2} \right\rceil )
                        + \ldots + \log ( \left\lceil \frac{n}{4} \right\rceil ) 
                        \tag*{(Taking half the terms)} \\
                        &\geq \left\lceil \frac{n}{4} \right\rceil \cdot 
                        \log ( \left\lceil \frac{n}{4} \right\rceil ) \tag*{(All cases 
                        are less than equal to original)} \\ 
                        &\geq \frac{n}{4} \cdot \log(\frac{n}{4}) \tag*{($
                        \left\lceil \frac{n}{4} \right\rceil \geq 
                    \frac{n}{4}$)}\\
                        &= \frac{n}{4} \log n - 
                        \frac{n}{4} \log 4 \\
                        &\geq \frac{n}{4}  \log n - 
                         \frac{n}{8} \log n \tag*{
                        (Since $n \geq 16$)}  \\
                        &= \log n \cdot \left( \frac{n}{4}  - 
                        \frac{n}{8}\right) \\
                        &=  \frac{n}{8} \log n \\
                        &= \frac{1}{8} n \cdot \log n \\
                        &= c \cdot n \log n \\
                        &\in \Omega (n \log n)
                \end{align*}
                Thus we have proven that $\log (\left\lceil \frac{n}{2} \right\rceil !)$
                is $\Omega (n \cdot \log n)$.

                Since both the upper bound and the lower bound are the 
                same, this algorithm is $\Theta (n \cdot \log n)$

                \qedsymbol
        \end{enumerate} 
        \newpage 
    \item \textbf{Average-case analysis.}
        \begin{enumerate}[label=(\alph*)]
            \item Find a set of inputs $\mathcal{I}_{has\_even, n}$ for 
                which $Avg_{has\_even} (n)$ is $\Theta (1)$.
                
                Let $ \mathcal{I}_{has\_even, n} = \{ \text{Input lists 
                $A$} \mid \text{len($A$)} = n \land (\exists k \in \mathbb{Z}, 
                A [0] = 2k )\}$. Since the first index is always even, 
                has\_even will always return on the first index, which makes 
                the algorithm have a constant runtime, which means 
                $Avg_ {has\_even} (n) \in \Theta (1)$. 
            \item Find a set of inputs $\mathcal{I}_{has\_even, n}$ for 
                which $Avg_{has\_even} (n)$ is $\Theta (n)$.
                
                Let $ \mathcal{I}_ {has\_even, n} = \{ \text{Input lists 
                $A$} \mid \text{len($A$)} = n \land (\forall i \in 
                \text{range($n$)}, \exists k \in \mathbb{Z}, 
            A[i] = 2k + 1)  \}$. Since all numbers in the lists are 
            odd, the algorithm loops for $n$ iterations on any of the 
            input lists described in this set before returning 
            False, which is $\Theta (n)$.
        \item $S_n = \{ \text{Input lists } numbers \text{ to } has\_even 
            \mid \forall i \in \text{range($n$)}, 1 \leq numbers[i] \leq n\}$

            Consider the set of allowable inputs $\mathcal{I}_{has\_even, n} = 
            S_n$. Give an expression for $| \mathcal{I}_{has\_even, n}|$. 
                
            Since we can have repeats within the pairs, we know that 
            each index of a list would have $n$ choices, and the 
            list is $n$ length. 
            $$n \times n \times \ldots \times n \times n \text{ ($n$ times)}$$
            So $| \mathcal{I}_{has\_even, n} | = n^n $. 
        \item Calculate an exact expression for $Avg_{has\_even} (n)$ for 
            the set of allowable inputs $\mathcal{I}_n = S_n$, 
            where $S_n$ is as defined in the previous part. 
    
            To find the average-case for has\_even, we need to take the 
            weighted average of the runtimes for the set of allowable inputs
            $S_n$. For every input list in $S_n$ where the first even 
            number is at index $i$, the algorithm runs exactly 
            $i + 1$ times. When there does not exist an even number, 
            there are $(n+1)$ steps of the loop and the return statement. 
            Substituing these steps into the weighted average we have
            \begin{align*}
                Avg_{has\_even} (n) &= \frac{1}{|S_n|} \sum_{i=0}^{n-1} 
                \sum_{lst \in S_n} (1+i) \tag*{(Where the first even number is at $i$)} 
            \end{align*}
            We know from part (c) $|S_n| = n^n$. Consider a list where 
            the first even number is at $i$, then at that index, there would 
            be $\left\lfloor \frac{n}{2} \right\rfloor $ possibilities where 
            $lst[i]$ is even. Counting up all the previous indices, then 
            the possibilties where all indices up to and including $i$ are even 
            can be expressed as $\left\lfloor \frac{n}{2} \right\rfloor ^i$. We 
            also have to consider the indices after $i$, which could be either 
            even or odd, so $n^{n-i-1}$ possibilities. 

            Let $a = \left\lceil \frac{n}{2} \right\rceil $,
            let $b = \left\lfloor \frac{n}{2} \right\rfloor$, and
            $r = \frac{a}{n}$ Then we have 
            \begin{align*}
                Avg_{has\_even} (n) &= \frac{1}{n^n} \left[\sum_{i=0}^{n-1} 
                    (i + 1)\left\lceil
                    \frac{n}{2} \right\rceil^{i} \left\lfloor 
                \frac{n}{2}\right\rfloor
        n^{n-i-1} \right] + \frac{1}{n^n}(n+1)\left\lceil  \frac{n}{2} \right\rceil^n \\
                                    &= \left[
                                    \sum_{i'=1}^{n} i' 
                                    a ^{i' -1}
                                    b \frac{n^{n-i'}}{n^n} \right]
                                + \frac{a^n}{n^n} (n+1) \tag*{($i' = i+1$)}\\
                                    &= \left[ \sum_{i'=1}^{n} i' \frac{b}{a} 
                                 \left( \frac{a}{n}\right)^ {i'}\right] 
                                    + r^n (n + 1) \\
                                &= \frac{b}{a}\cdot
                                \left[ \sum_{i' = 1}^{n} i' r^ {i'} \right] 
                                + r^n (n+1)
            \end{align*}
        We will now split 
            the calculations to the cases where $n$ is even and $n$ is odd.
            \newpage
            \textbf{Case 1.} Assume $n = 2k$, where $k \in \mathbb{Z}$.
            
            Then we know that $\left\lceil \frac{2k}{2} \right\rceil = 
            \left\lfloor \frac{2k}{2} \right\rfloor= \frac{2k}{2} = a = b$. We also 
            know that $r = \frac{k}{2k} = \frac{1}{2}$. 
            We will simplify the first term then the second term before combining:
            \begin{align*}
             1 \cdot \sum_{i=1}^{n} i' \left( \frac{1}{2}\right)^ {i'} &= 
                \sum_{i'=0}^{n} i'\left( \frac{1}{2} \right)^{i'} \tag*{(Term at $i=0$ is 0)} \\
                   &= \frac{n (\frac{1}{2})^ {n+1}}{ (\frac{1}{2})-1)} - 
                       \frac{\frac{1}{2} ( (\frac{1}{2})^n - 1)}{ 
                       (\frac{1}{2} - 1)^2} \tag*{(Closed-form)}\\
                   &= - \frac{n}{2^{n}} - 2 \left( \left(\frac{1}{2} \right)^n - 1
                       \right)\\
                    \frac{ (n+1) (\frac{n}{2})^n}{n^n} &= \frac{ (n+1) n^n}{ 2^n n^n } \\
                                                       &= \frac{ (n+1)}{2^n} \\
                    Avg_ {has\_even (even \: n)} (n)&= 
                - \frac{n}{2^n} - 2 \left( \left( \frac{1}{2} \right)^n - 1 \right) 
                    + \frac{ (n+1)}{2^n} \\ &= \frac{1}{2^n} - 2(\frac{1}{2^n} - 1)  \\
                                            &= 2 - \frac{1}{2^n}
            \end{align*}
            So we have $2 - \frac{1}{2^n}$ steps for 
            $Avg_ {has\_even} (n)$ when $n$ is even.

            \textbf{Case 2.} Assume $n = 2k + 1$, where $k \in \mathbb{Z}$.
            
            Then we know $a = \frac{n + 1}{2}$ ,and 
            $b = \frac{n - 1}{2}$. 
            We will simplify the first term before combining:
            \begin{align*}
                \frac{b}{a} \sum_{i'=1}^{n} i'r^{i'}  &= 
                \frac{b}{a} \sum_{i'=0}^{n} i'  r^{i'} \tag*{(Term 
                at $i'=0$ is 0)} \\
                &= \frac{n-1}{n+1} \left( \frac{n r^{n+1}}{ r - 1} -
                  \frac{r (  r^n - 1)}{
                 ( r - 1)^2} \right) \\
            Avg_ {has\_even (odd \: n)} (n) &= 
            (n+1)r^n  + 
             \frac{n-1}{n+1} \left( \frac{n r^{n+1}}{ r - 1} -
                  \frac{r (  r^n - 1)}{
                 ( r - 1)^2} \right) \\
            \end{align*}
            Thus the exact expression would be
            \begin{equation}
                Avg_ {has\_even} (n) = 
                \begin{cases}
                    2 - \frac{1}{2^n}, \:& \text{when $n$ is even}\\
            (n+1)r^n  + 
             \frac{n-1}{n+1} \left( \frac{n r^{n+1}}{ r - 1} -
                  \frac{r (  r^n - 1)}{
                 ( r - 1)^2} \right),
                 \: & \text{when $n$ is odd} 
                \end{cases}
            \end{equation}
            where 
            $r = \frac{a}{n} = \frac{\left\lceil \frac{n}{2} \right\rceil }{
            n}$.

    \end{enumerate}
\end{enumerate}

\end{document}
