% CSC165 Winter 2023: Problem Set 0
% Template to use to complete Problem Set 0.
% Note that using this template is *optional*: it provides a nice foundation
% for getting started with LaTeX, but you aren't required to use it!
% If you are using Overleaf, you'll want to upload this file to your account.
% Before modifying this file, we recommend trying to compile it as-is
% to see what the basic template gives.

\documentclass[12pt]{article}

\usepackage{amssymb}
\usepackage{amsmath}
\usepackage[margin=2.5cm]{geometry}

% If you want to use this package, make sure to download it from the LaTeX
% help page on Quercus (and copy to Overleaf, if using)! Then remove the
% comment indicator % that starts the next line.
% \usepackage{csc}

% Document metadata
\title{CSC165H1: Problem Set 0}
\author{Michael Zi Xi Zhou}
\date{2023-01-13}


% Document starts here
\begin{document}
\maketitle

\section*{My Courses}

% Use the environment "itemize" to create an *unordered* list.

\begin{itemize}
    \item CJH332H1, Cellular and Molecular Neurobiology of the Synapse, Jimmy Fraigne
    \item CSC148H1, Introduction to Computer Science, Jonathan Calver 
    \item CSC165H1, Mathematical Expression and Reasoning for Computer Science, Thomas Fairgrieve
    \item HMB200H1, Introduction to Neuroscience, Paul Whissell 
    \item IMM250H1, The Immune System and Infectious Disease, Matthew Buechler
\end{itemize}

\section*{Set notation}

% Fill in the following:

\[
S_1 \cap S_2 = \{0, 1, 4, 5, 6, 9, 10, 11, 14\}
\]


\section*{A truth table}
\begin{center}
\begin{tabular}{| c c c c|}
\hline
    $p$ & $q$ & $r$ & $(p \lor q) \Rightarrow (p \Leftrightarrow r)$ \\
    \hline
    True & True & True & True \\
    True & True & False & False \\
    True & False & True & True \\
    True & False & False & False \\
    False & True & True & False \\
    False & True & False & True \\
    False & False & True & True \\
    False & False & False & True \\
    \hline
\end{tabular}
\end{center}

\newpage
\section*{A calculation}

We will use the arithmetic series formula $\sum_{i=1}^{n-1} (di + k) = 
nk + \frac{dn (n-1)}{2}$ to simplify the equation. 
\begin{align*}
    \sum_{i=0}^{n-1} (2i + 3) &= 3n + \frac{2n(n-1)}{2} \\
                              &= 3n + \frac{2n^2 - 2n}{2} \\
                              &= 3n + \frac{2n^2}{2} - \frac{2n}{2} \\
                              &= 3n + n^2 - n  \\
                              &= n^2 + 2n \\ 
\end{align*}
Finally, we will find the smallest positive integer $n$ such that 
$n^2 + 2n \geq 165$ using the quadratric formula. 
\begin{align*}
    n^2 + 2n &\geq 165 \\
    n^2 + 2n - 165 &\geq 0 \\
    n &\geq \frac{-2 \pm \sqrt{2^2 - 4 \cdot 1 \cdot -165}}{2 \cdot 1} \\
    n &\geq \frac{-2 \pm \sqrt{664}}{2} \\
    n &\geq \frac{-2 \pm 2\sqrt{166}}{2} \\
    n &\geq {-1 + \sqrt{166}} \\
    n &\approx 11.884\\
    n &= 12 \\
\end{align*}
We can remove the negative case and round up to the positive integer 
$12$ since $n \in \mathbb{Z}^+$, and we take the lowest possible $n$, 
which is $n = {-1 + \sqrt{166}}$, which is rounded to $n = 12$. 

\end{document}
