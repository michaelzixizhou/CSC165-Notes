\documentclass{article}
\usepackage[margin=1in]{geometry}
\usepackage{lipsum}
\usepackage{parskip}
\usepackage{amsfonts}
\usepackage{amsmath}
\usepackage{amssymb} 
\usepackage [english]{babel}
\usepackage [autostyle, english = american]{csquotes}
\MakeOuterQuote{"}
\usepackage{enumitem}
\title{CSC165H1: Problem Set 3}
\author{Michael Zhou}
\date{2023-03-13}

\newcounter{qcount}
\newcommand\q{\stepcounter{qcount} \textbf{Question \theqcount}. }
\newcommand\qedsymbol{\hfill$\blacksquare$}              

\begin{document}
\maketitle
\q \textbf{Number representation.}

For all $n \in \mathbb{N}$ and $k \in \mathbb{Z}^+$, we define $C (n, k)$ 
to be:
$$\exists a_1, \ldots, a_k \in \mathbb{N}, (\forall i \in \mathbb{Z}^+, 
1 \leq i \leq k \Rightarrow a_i \leq i) \land (n = \sum_{i=1}^{k} a_i 
\cdot i!)$$
Prove using induction that $\forall n \in \mathbb{N}, \forall k \in 
\mathbb{Z}^+, n < (k+1)! \Rightarrow C (n, k)$.

\textit{Proof.} We define the predicate $P (k):
`` \forall n \in \mathbb{N}, n < (k +1 )! \Rightarrow C (n, k) "
$, where $k \in \mathbb{Z}^+$. 

\textbf{Base case.} Let $k = 1$. We want to prove $P(1)$. Let $n \in \mathbb{N}$. 
Assume $n < 2!$. Since $k = 1$, we know $n = \sum_{i=1}^{1} a_i 
\cdot i!$ = $a_1$. And $a_1$ can be either $1$ or $0$, and both cases will result in 
$n< 2$. Thus the base case is True for $P(1)$.

\textbf{Inductive step.} Let $x \in \mathbb{Z}^+$. Assume $P(x)$ is True, such that 
$\forall n \in \mathbb{N}, n < (x + 1)! \Rightarrow C (n , x)$. We want to 
prove that $P(x+1)$ is True, such that $\forall n \in \mathbb{N}, n < 
(x + 2)! \Rightarrow C (n, x + 1)$. 

Let $n \in \mathbb{N}$. Assume $n < (x + 2)!$. We want to prove that $
C(n, x + 1) $. Since we know that $n < (x + 1)!$ or $n \geq (x + 1)!$, we 
can divide the proof into two cases:

\textbf{Case 1.} Assume $n < ( x + 1)!$.

Then we know $C (n , x)$ is True by the induction hypothesis. So we know that
$\exists a_1, \ldots, a_x \in \mathbb{N}, (\forall i \in \mathbb{Z}^+, 
1 \leq i \leq x \Rightarrow a_i \leq i)$, and $n = \sum_{i=1}^{x} a_i \cdot 
i!$. In other words, $a_1, \ldots, a_k$ satisfy $n = \sum_{i=1}^{x} a_i \cdot 
i!$. 

Let $a'_1 = a_1, a'_2 = a_2, \ldots, a'_x = a_x, a'_ {x + 1} = 0$. Then 
we have 
\begin{align*}
    n  &= a_1 \cdot 1! + a_2 \cdot 2! + \ldots + a_x \cdot x! + 0 \cdot (x+1)!\\
       &= a'_1 \cdot 1! + a'_2 \cdot 2! + \ldots + a'_x \cdot x! + a'_ {x+1} 
       \cdot (x+1)!\\
     &= \sum_{i=1}^{x+1} a_i \cdot i!
\end{align*}
We know that $a'_ {x+1} = 0$ satisfies the first part of the predicate 
since $0 < i$ for all positive integers $i$.
So $C (n, x + 1)$ is True for $n < (x + 1)!$.

\textbf{Case 2.} Assume $n \geq (x + 1)!$.

By the assumptions of the statement we want to prove, 
we know that $n$ can be represented using the Quotient Remainder Theorem: 
$\exists q, r \in \mathbb{Z}$ such that all $n$ between $(x+1)! \leq n < 
(x+2)!$ can be written as $q (x+1)! + r$. We know that $0 \leq r < (x+1)!$ by the 
definition of theorem, so $r$ can already be represented by $P(x)$ using 
the proof in Case 1 such that there exists $a_1, \ldots, a_x$ that satisfy 
$r = \sum_{i=1}^{x} a_i \cdot i!$.
\newpage
We also know that in order for $n$ to be less than $(x+2)!$, 
$q$ has to be less than $(x+2)$, or in other words, $q \leq (x+1)$. Let 
$q = a_ {x+1}$, then we get 
\begin{align*}
    n = q \cdot (x+1)! + r \\
    n = a_ {x+1}\cdot (x+1)! + \sum_{i=1}^{x} a_i \cdot i!\\
    n = \sum_{i=1}^{x+1} a_i \cdot i! 
\end{align*}
We know that $a_ {x+1} = q$ satisfies the first part of the statement 
since $q \leq (x+1)$. 

So $C (n , x+1)$ is True
by the assumption of $n < (x+2)!$. 

Thus $P(x+1)$ is True
by the assumption of $P(x)$, so we have proven the predicate $P(k): 
`` \forall n \in \mathbb{N}, n < (k+1)! \Rightarrow C (n, k)"$ where 
$k \in \mathbb{Z}^+$ by induction. 

\qedsymbol

\newpage
\q \textbf{Induction.}

For all $m, n \in \mathbb{N}$, let $A_m = \{ a \mid a \in \mathbb{N} \land 
a \leq m \} $ and $B_n = \{ b \mid b \in \mathbb{N} \land b \leq n \}$, 
and define $F_{m, n }$ to be:
$$\{ f : A_m \rightarrow B_n \mid [ \forall k, l \in A_m, k \leq l 
\Rightarrow f ( k) \leq f (l) ] \land f (m) = n  \}$$
For all $m, n \in \mathbb{N}$, define $P (m, n)$ to be:
$$ | F_ {m, n} | = \frac{ (m + n)!}{m! \cdot n!}$$
\begin{enumerate}[label = (\alph*)]
    \item Prove each following statement.
        \begin{enumerate}[label = (\roman*)]
            \item $\forall m \in \mathbb{N}, P (m, 0)$. 

                \textit{Proof.} Let $m \in \mathbb{N}$. We want to prove 
                that $P (m, 0)$ is True. 

                Since the codomain is a set that contains only 0, the size 
                of $F_ {m, n}$ would only be 1, since the only function that 
                would satisfy the conditions would be the one that mapped 
                all $m$ to 0. It would look like a horizontal line at 
                the origin, since $f(0), \ldots, f(m)$ must all be 
                natural numbers less than 
                or equal to 0.
                \begin{align*}
                    LHS &= |F_ {m, n}|\\ 
                        &= 1 \\
                    RHS &= \frac{ (m + 0)!}{m! \cdot 0!} \\
                      &= \frac{m!}{m!} \\
                      &= 1 \\
                    LHS &= RHS = 1
                \end{align*}
                Thus we have proven both sides of $P (m, 0)$ are equal, 
                so the statement is True for all $m \in \mathbb{N}$. 
                
                \qedsymbol
            \item $\forall n \in \mathbb{N}, P (0, n)$. 

                \textit{Proof.} Let $n \in \mathbb{N}$. We want to prove 
                that $P(0, n)$ is True. 
                
                Since the domain is a set that only contains 0, we would have 
                one function in $F_ {m, n}$ since $A_m = \{0\}$. The function 
                would only have $f (0) = n$.
                \begin{align*} 
                    LHS &= |F_ {m, n}| \\
                        &= 1 \\
                    RHS &= \frac{ (0 + n)!}{0! \cdot n!} \\
                        &= \frac{n!}{n!} \\
                        &= 1 \\
                    LHS &= RHS = 1
                \end{align*}
                Thus we have proven both sides of $P (0, n)$ are True, 
                so the statement is True for all $n \in \mathbb{N}$.

                \qedsymbol
                \newpage
            \item $\forall m, n \in \mathbb{N}, P (m, n+1) \land 
                P (m + 1, n) \Rightarrow P (m + 1, n + 1)$.

                \textit{Proof.} Let $m , n \in \mathbb{N}$. Assume 
                $P ( m , n + 1) \land P(m+1, n)$. We want to show that 
            $P (m +1, n+1)$, such that $| F_ {m + 1, n + 1}| = 
            \frac{ (m + n + 2)!}{ (m + 1)! (n + 1)!}$. 

                We know every function in $F_ {m , n +1}$ has $f (m) = n + 1$. 
                To convert these functions to members of $F_ {m + 1, n + 1}$, 
                we would have to add the assignment $f ( m + 1) = n + 1$. These 
                functions would include all cases where $f (m) = n + 1$. 
                
                We need to also do this for $F_ {m + 1, n}$, 
                and make $f (m + 1) = n +1$ for all of these functions. This new 
                set would have all cases where $f ( m) \leq n$.

                We are not introducing any new functions into these sets, 
                but instead modifying the existing functions so $f (m + 1) 
                = n+1$. These functions would all satisfy the statement 
                by the hypothesis.

                There would not 
                be any overlaps since the $F_ {m, n+1}$ set includes of all 
                cases where $f (m )= n + 1$ and the $F_ {m+1, n}$ group takes care 
                of all cases where $ f (m) \leq n$. 

                The cardinality 
                of $F_ {m + 1, n+ 1}$ would be the total amount of functions that 
                that can have their final term be $f (m+1) = n+1$, such
                that $f (m) \leq n +1$, which means the union of $F_ {m, n +1}$ and 
            $ F_ {m+1, n}$ would cover all cases where $f (m) \leq n +1$ and $f
            (m +1) = n +1$.

                In other words, $|F_ {m +
                1, n + 1}| = |F_{m, n+1}| + |F_ {m +1, n}|$.
                
                 Substituting into the equation, we have 
                \begin{align*}
                    LHS &= |F_ {m+1, n+1}| \\
                        &= \frac{ ( m + n + 2)!}{ (m + 1)! (n + 1)} \tag*{(By 
                        definition of $P(m, n)$)} \\
                    RHS &= |F_ {m, n + 1}| + |F _ {m + 1, n}| \\
                        &= \frac{ (m + n + 1)!}{m! (n+1)!} + \frac{
                        (m + n +1)!}{ (m + 1)! n!} \tag*{ (By assumption)} \\
                        &= \frac{ (m + n + 1)! (m +1)}{ (m + 1)! (n +1)!} + 
                        \frac{ (m + n +1)! (n + 1)}{ (m + 1)! (n + 1)!} \\
                        &= \frac{ (m + n + 1)! ( (m + 1) + (n + 1))}{ 
                        ( m + 1)! (n + 1)!} \\
                        &= \frac{ ( m + n + 1)! ( m + n + 2)}{ (m + 1)! (n + 1)!} \\
                        &= \frac{ (m + n + 2)!}{ (m + 1)! (n + 1)!} \\
                    LHS &= RHS
                \end{align*}
                Both sides are equal assuming $P (m, n +1)$ and $P ( m + 1, n)$, 
                thus $P (m + 1, n + 1)$ is True.
                
                \qedsymbol
        \end{enumerate}
    \newpage
    \item Prove, using results from part (a), that $P (1,1) \land P (2,2)$.
        
        \textit{Proof.} We want to prove $P (1, 1) \land P (2, 2)$. 
        Let $m, n \in \mathbb{N}$. By part (a), we know that $P(0, n)$ 
        and $P (m, 0)$ are both True. If we let $m = 1$ and $n = 1$, 
        then $P(0, 1)$ and $P (1, 0)$ are both True. Similarly for 
        $P (0, 2)$ and $P (2, 0)$.

        By part (a), we know 
        that if $P(m, n+1) \land P(m+1, n)$, then $ P (m +1. n+1)$ would 
        be also be True. If we let $m = 0$ and $n = 0$, we get 
        $P (0, 1) \land P (1, 0)$, which we know are both True, then we know 
        that $P (1, 1)$ is also True. 

        We can also do the same for 
        $P (2, 0)$ and $P (1, 1)$ by letting $m = 1$ and $n = 0$ and proving 
        $P (2, 1)$. And the same goes for $P (0, 2)$ and $P (1, 1)$ by 
        letting $m = 0$, $n = 1$, proving $P(1, 2)$. 

        Finally, using $m = 1$ and $n = 1$, we have $ P (2, 1) 
        \land (1, 2) \Rightarrow P(2, 2)$. 

        Thus both $P (1, 1)$ and $P (2, 2)$ are True by the results 
        from part (a).

        \qedsymbol

    \item For each $t \in \mathbb{N}$, define $Q (t)$ to be $\forall m, n 
        \in \mathbb{N}, m + n = t \Rightarrow P (m ,n)$. Prove using induction 
        and the results from part (a), that: $\forall t \in \mathbb{N}, 
        Q(t)$ 

        \textit{Proof.} Let $t \in \mathbb{N}$. We want to prove $Q (t)$
        such that for all $m, n \in \mathbb{N}$ if $m + n = t$, then 
        $P (m, n)$. 

        \textbf{Base case.} Let $t = 0$. Assume that $t = m + n$. We want 
        to show that $P(m, n)$ is True. Since $m, n$ are both natural 
        numbers, they must both be 0 since $m + n = 0$. Then $P (0, 0)$ 
        is True by the results in part (a).

        \textbf{Inductive step.} Let $k \in \mathbb{N}$. Assume that 
        $Q (k)$ is True such that $m_1 + n_1 = k \Rightarrow P(m_1, n_1)$. We 
        want to prove that $Q (k+1)$ is True, such that $m_2 + n_2 = 
        k + 1 \Rightarrow P (m_2, n_2)$. We will use a proof by cases.

        We know that if $m_2 + n_2 = k + 1$, either $m_2 = m_1 + 1 \land 
        n_2 = n_1$, or $m_2 = m_1 \land n_2 = n_1 + 1$. 
        
        \textbf{Case 1.} Assume $m_2 = m_1 + 1 \land n_2 = n_1$. 

        Using the induction hypothesis, we have $P (m_1, n_1) = P (m_2 - 1, 
        n_2)$. We also have $P (m_2, n_2 - 1)$ by rearranging 
        $k = m_1 + n_1 = m_2 - 1 + n_2 = m_2 + n_2 - 1$. By part (a), 
        we can conclude that since $ P (m_2 -1 , n_2) \land P (m_2, n_2 -1)$, 
        that $P (m_2, n_2)$ is True in this case.

        \textbf{Case 2.} Assume $m_2 = m_1 \land n_2 = n_1 + 1$.

        Following the same procedure as Case 1, we have 
        $P (m_1, n_1) = P (m_2, n_2 - 1 )$, and rearranging $k = m_2 + n_2 - 1$ 
        we get $P ( m_2 -1, n_2)$. By part (a), we can conclude that 
        that since $P (m_2 -1, n_2) \land P( m_2, n_2 -1)$, that 
        $P (m_2, n_2)$ is also True in this case.

        Thus we have proven that $P (m_2, n_2)$ is True in both cases
        by the assumption that $m_2 + n_2 = k + 1$, such that $Q(k+1)$ is 
        True by the assumption that $Q (k)$ is True. We have 
        proven $Q (t)$ by induction.

        \qedsymbol

    \item Prove, using results form part (c), that: $\forall m, n \in \mathbb{N}, 
        P (m, n)$.

        \textit{Proof.} Let $m, n \in \mathbb{N}$. We want to prove 
        that $P ( m , n)$. 

        Let $t \in \mathbb{N}$. From part (c), we know that if $m + n = t$ 
        then $P ( m, n)$ is True. 

        If you add two natural numbers, the result will always be a 
        natural number, so there will always be a natural number $t$ such 
        that $m + n = t$. 

        Thus, by part (c), $P (m, n)$ is True for any natural numbers 
        $m, n$. 

        \qedsymbol


\end{enumerate}

\newpage
\q \textbf{Asymptotic notation.}
\begin{enumerate}[label = (\alph*)]  
    \item Prove or disprove that $n^n \in \mathcal{O} (n!)$.

        \textit{Translation.} $\exists c, n_0 \in \mathbb{R}^+, 
        \forall n \in \mathbb{N}, n \geq n_0 \Rightarrow n^n \leq 
        c \cdot n!$

        \textit{Proof.} 
        Let $n \in \mathbb{N}$. 
        We want to prove 
        that $n^n \notin \mathcal{O} (n!)$. We will use a proof by contradiction. 
            
        Let $c, n_0$ be arbitrary positive real numbers. 
        Assume that $n^n \in \mathcal{O} (n!)$, such that if $n \geq n_0 
        $ then 
        \begin{align*}
            n^n &\leq c \cdot n! \\
            \frac{n^n}{n!} &\leq c
        \end{align*} 
        Then there would exist a $c$ that is always greater than 
        $\frac{n^n}{n!}$ by the assumption that $n^n \in \mathcal{O} (n!)$. 
        
        We know that $n^n$ is larger than $n!$ since if we line up 
        all values, the values of $n^n$ are all greater than $n!$ after 
        the first term, so the ratio grows to infinity:
        \begin{align*}
            n^n &= n \cdot n \cdot \ldots \cdot n \cdot n \tag*{($n$ terms)} \\
            n! &= n \cdot (n - 1) \cdot \ldots \cdot 2 \cdot 1 \tag*{($n$ terms)}
        \end{align*}
        But we know that there does not exist a largest natural number, 
        since the set is infinite. Thus, there cannot exist a $c$ that 
        always satisfies this inequality, creating a contradiction.
        
        \qedsymbol
        
    \item Prove that if $a, b \in \mathbb{R}$ and $b > 0$, then 
        $ (n + a)^b \in \Theta (n^b)$.

        \textit{Proof.} Let $a, b \in \mathbb{R}$. Assume that 
        $b > 0$. We want to prove that $ (n+a)^b \in \Theta (n^b)$. 
        
        Expanding on the definition of $\Theta (n^b)$, we have 
        \begin{align*}
            \exists c_1, c_2, n_0 \in \mathbb{R}^+, \forall n \in \mathbb{N}, 
            n \geq n_0 \Rightarrow
            c_1 n^b\leq (n + a)^b \leq c_2 n^b 
        \end{align*}
    Let $c_1 = (\frac{1}{2})^b$, $c_2 = (\frac{3}{2})^b$, $n_0 = 2 | a |$. Assume
    $n \geq n_0$. We want to show that $ c_1 n^b \leq (n + a)^b \leq c_2 n^b$.
    Solving for the left side we get 
    \begin{align*}
        (n+a)^b &= \left(\frac{n}{2} + \frac{n}{2} + a\right)^b  \\
                &\geq \left(\frac{n}{2} + |a| + a \right) ^b \tag*{(Since $n > 2|a|$)}\\
                &\geq \left( \frac{n}{2} \right)^b \tag*{(Since $|a| + a \geq 0$)} \\
                &\geq \left(\frac{1}{2}\right)^b \cdot n^b \\
                &\geq c_1 n^b
    \end{align*}
    Solving for the right side we get 
    \begin{align*}
        (n+a)^b &\leq \left(n + \frac{n}{2}\right)^b \tag*{(Since $ n \geq 2 |a|$)} \\
                &\leq \left(\frac{3n}{2}\right)^b \\
                &\leq \left(\frac{3}{2}\right)^b \cdot n^b \\
                &\leq c_2 n^b 
    \end{align*}
    Thus we have proven $c_1 n^b \leq (n + a)^b \leq c_2 n^b$ assuming 
    $n \geq n_0$ and $b > 0$. 

    \qedsymbol

\end{enumerate}

\newpage
\q \textbf{More asymptotic notation.}
\begin{enumerate}[label = (\alph*)]
    \item Prove or disprove that: if $f : \mathbb{N} \rightarrow \mathbb{R}^
        {\geq 0}, k \in \mathbb{R}^+$ and $f (n) \in \mathcal{O} (n^k)$, then 
        log$_2 (f (n)) \in \mathcal{O} (\log_2n)$.

        \textit{Proof.} Let $k \in \mathbb{R}^+$. We want to show that 
        if $f : \mathbb{N} \rightarrow \mathbb{R}^ {\geq 0}$ and 
        $f (n) \in \mathcal{O} (n^k)$, then $\log_2 (f(n)) \in \mathcal{O}
        (\log_2 n)$. Let $n_0$ be an arbitrary positive real number. Let 
        $c = k + 1$. Let $n_0 = c$. Let $n \in \mathbb{N}$. Then 
        by our assumption and the definition of Big-Oh, if $n \geq n_0$, we have 
        \begin{align*}
            f (n) &\in \mathcal{O} (n ^k) \\
            f (n) &\leq c \cdot n^k  \\
            \log_2 ( f (n)) &\leq \log_2 (c\cdot n^k) \\
            \log_2 (f (n )) &\leq \log_2 (c) + \log_2 (n^k) \tag*{(Using log identity)} \\
            \log_2 (f (n)) &\leq \log_2 (n) + \log_2 (n^k) \tag*{(Since $n \geq c$)} \\
            \log_2 (f (n)) &\leq log_2(n) + k \cdot \log_2 (n) \\
            \log_2 (f (n)) &\leq (k+1) \cdot \log_2 (n) \\
            \log_2 (f (n)) &\leq c \cdot \log_2 (n) \\ 
            \log_2 (f (n)) &\in \mathcal{O} (\log_2 (n))
        \end{align*}
        Thus we have proven the $\log_2 (f (n)) \in \mathcal{O} \log_2 n$ 
        by the assumption that $f (n) \in \mathcal{O} (n^k)$. 

        \qedsymbol

    \item Prove that: if $f_1, f_2, g_1, g_2: \mathbb{N} \rightarrow \mathbb{R}^ {\geq 0}
        , f_1 \in \mathcal{O} (g_1)$, and $f_2 \in \mathcal{O} (g_2)$, then 
        $f_1 + f_2 \in \mathcal{O} (max (g_1, g_2))$. Here, $ (f_1 + f_2) (n)
        = f_1 (n) + f_2 (n)$ and max$ (g_1, g_2) (n) =$ max$(g_1 (n), g_2 (n))$.
        
        \textit{Proof.} Assume that $f_1 , f_2, g_1, g_2: \mathbb{N} \rightarrow 
        \mathbb{R}^ {\geq 0}$, $f_1 \in \mathcal{O} (g_1)$, and 
        $f_2 \in \mathcal{O} (g_2)$. We want to show that 
        $f_1 + f_2 \in \mathcal{O} (\text{max}(g_1, g_2))$. 

        Using the definition of Big-Oh, we can express the hypothesis
        as $\exists c_1, c_2 , n_1 , n_2\in \mathbb{R}^+, \forall n \in 
        \mathbb{N}, (n \geq n_1 \Rightarrow f_1  \leq c_1 \cdot 
        g_1) \land (n \geq n_2 \Rightarrow f_2 \leq c_2 \cdot 
        g_2)$.
            
        Let $c_3 = c_1 + c_2$, let $n_3 = \text{max} (n_1, n_2)$. Then 
        by our assumptions, we know that $f_1 \leq c_1 \cdot g_1$ and 
        $f_2 \leq c_2 \cdot g_2$. Adding these together we have 
        \begin{align*}
            f_1 + f_2 &\leq c_1 \cdot g_1 + c_2 \cdot g_2 \\
             f_1 + f_2&\leq c_1 \cdot \text{max} (g_1, g_2) + c_2 \cdot
                \text{max} (g_1, g_2) \tag*{(max($g_1, g_2$) $\geq g_1$, same for $g_2$)}\\
                 f_1 + f_2     &\leq (c_1 + c_2) \cdot \text{max}  (g_1, g_2) \\
               f_1 + f_2       &\leq c_3\cdot \text{max} (g_1, g_2) \\
               f_1 + f_2 &\in \mathcal{O} (\text{max} (g_1, g_2)) \tag*{($n \geq n_3$)}
        \end{align*}
        Thus we have proven $f_1 + f_2 \in \mathcal{O} (\text{max} (g_1, g_2))$ 
    assuming that $f_1 \in \mathcal{O}(g_1)$ and $f_2 \in \mathcal{O} (g_2)$. 

        \qedsymbol

\end{enumerate}

\end{document}
