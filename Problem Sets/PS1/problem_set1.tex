\documentclass{article}
\usepackage[margin=1in]{geometry}
\usepackage{lipsum}
\usepackage{parskip}
\usepackage{amsfonts}
\usepackage{amsmath}
\usepackage{amssymb} 
\usepackage [english]{babel}
\usepackage [autostyle, english = american]{csquotes}
\MakeOuterQuote{"}
\usepackage{enumitem}
\title{CSC165H1: Problem Set 1}
\author{Michael Zi Xi Zhou}
\date{2023-01-28}

\newcounter{qcount}
\newcommand\q{\stepcounter{qcount} \textbf{Question \theqcount}. }

\begin{document}
\maketitle

\q \textbf{Propositional formulas}

\begin{enumerate}[label= (\alph*)]
    \item $ (\neg p \Leftrightarrow q) \Rightarrow q$ 
        \begin{enumerate}[label=(\roman*)]
            \item Truth table. 
                \begin{center}
                    \begin{tabular}{| c c c |}
                \hline 
                $p$ & $q$ & $(\neg p \Leftrightarrow q) \Rightarrow q$ \\
                \hline 
                True & True & True \\
                True & False & False \\
                False & True & True \\
                False & False & True \\
                \hline
                    \end{tabular}
                \end{center}
            \item Logically equivalent formula.
                \begin{align*}
                    (\neg p \Leftrightarrow q) &\Rightarrow q \\
                    ( ( \neg p \Rightarrow q) \land ( q \Rightarrow \neg p)) 
                                               &\Rightarrow q \\
                    ( ( p \lor q ) \land (\neg q \lor \neg p)) &\Rightarrow q \\
                    \neg ( ( p \lor q ) \land (\neg q \lor \neg p)) &\lor q \\
                    \neg(p \lor q) \lor \neg ( \neg q \lor \neg p) &\lor q 
                \end{align*}
                $$\boxed{ (\neg p \land \neg q) \lor (q \land p) \lor q}$$
        \end{enumerate}
    \item $ (p \Rightarrow (q \Rightarrow r)) \Rightarrow ( (p \Rightarrow q) \Rightarrow r)$
        \begin{enumerate}[label= (\roman*)]
            \item Truth table.
                \begin{center}
                    \begin{tabular}{|c c c c|}                   
                \hline
                $p $ & $q$ & $r$ & $(p \Rightarrow (q \Rightarrow r)) \Rightarrow ( (p \Rightarrow q) \Rightarrow r)$\\
                \hline
                True & True & True & True \\
                True & True & False & True \\
                True & False & True & True \\
                True & False & False & True \\
                False & True & True & True \\ 
                False & True & False & False \\
                False & False & True & True \\
                False & False & False & False \\
                \hline                      
                    \end{tabular}
                \end{center}
            \item Logically equivalent formula.
                \begin{align*}
                    (p \Rightarrow (q \Rightarrow r)) &\Rightarrow ( (p \Rightarrow q) \Rightarrow r) \\
                    \neg(\neg p \lor (\neg q \lor r)) &\lor (\neg (\neg p \lor q) \lor r) \\
                    (p \land \neg (\neg q \lor r)) &\lor ((p \land \neg q) \lor r) 
                \end{align*}
                $$\boxed{(p \land q \land \neg r) \lor ((p \land \neg q )
                \lor r)}$$
        \end{enumerate}
\end{enumerate}

\newpage
\q \textbf{Translating statements.}
\begin{enumerate}[label=(\alph*)]
    \item $\forall s \in S, \: \forall c \in C, \:  CS(s) \Rightarrow \neg Fail(s, c)$ 
    \item $\exists c \in C, \: \forall s \in S, \: CS(s) \Rightarrow Study(s, c)$
    \item $\exists s \in S, \: \forall c \in C, \: CS(s) \land \neg Study(s, c)$
    \item $\forall c \in C, \: \exists s_1 \in S, \: \forall s_2 \in S, \:
        Study(s_1, c) \land (Study(s_2, c) \Rightarrow s_1 = s_2)$
    \item $\forall s \in S, \: \exists c_1, c_2 \in C, \: \forall c_3 \in C, \:
        Study(s, c_1) \land Study(s, c_2) \land c_1 \neq c_2 \land (Study (s, c_3)
        \Rightarrow (c_3 = c_1 \lor c_3 = c_2)) $
\end{enumerate}

\newpage
\q \textbf{Choosing a univserse and predicates}
\begin{align*}
    \forall x \in \mathbb{N}, P(x, 165) \lor P(x, 0) \tag*{(Statement 1)} \\
    (\forall x \in \mathbb{N}, P (x, 165)) \Rightarrow (\exists x \in \mathbb{N}, 
    P(x, 0 )) \tag*{(Statement 2)}
\end{align*}
\begin{enumerate}[label=(\alph*)]
    \item (Statement 1) is False and (Statement 2) is True when both 
terms of (Statement 1) are False --- in this case, (Statement 2) will always be 
True due to the vacuous truth cases for implication operators. So, we can define
the binary predicate $P$ as the following:
$$\boxed{P(a, b) : ``b \mid a," \: \text{where } a, b \in \mathbb{N}} $$
Then (Statement 1) would be False since 165 does not divide all natural 
numbers, and 0 does not divide all natural numbers. And (Statement 2) 
would be True since the hypothesis is False from the first term of (Statement 1).
    \item For both statements to be True, only one term of (Statement 1) has 
        to be True due to the OR operator, and either the hypothesis of 
        (Statement 2) has to be False, or both terms have to be True. So, we 
        can define the binary predicate $P$ as the following:
        $$\boxed{P(a, b) : ``a \mid b," \: \text{where } a, b \in \mathbb{N}}$$
        Then (Statement 1) is True since all natural numbers divide 0 ($0 \cdot x = 0$), 
        and (Statement 2) is True since the hypothesis is False (not all natural numbers 
        divide 165).
    \item \begin{align*}
        \forall x \in S, \Big( \exists y \in T, P(x, y) \Big) \Rightarrow Q(x) \tag*{(Statement 3)}\\
        \forall x \in S, Q(x) \Rightarrow \Big( \forall y \in T, P(x,y) \Big) \tag*{(Statement 4)}
    \end{align*}
    Let's choose $Q$ as the False predicate use it to find $P$. 
    We will first define the sets $S$ and $T$:
    \begin{align*}
        &\boxed{S = \{ x \mid x \in \mathbb{N} \text{ and } 0 \leq x \leq 10 \}} \\
        &\boxed{T = \{ y \mid y \in \mathbb{N} \text{ and } 5 \mid y \}}
    \end{align*}
    If $Q$ is False for $\forall x \in S$, then we can define a predicate $P$ that is 
    True for at least one $y \in T$, but False for all $y \in T$. This will make (Statement 3)
    False and (Statement 4) True. We will now define a binary predicate $P$ 
    that is True for $\forall x \in S$ and $\exists y \in T$, but False for 
    $\forall x \in S$ and $\forall y \in T$, and a unary predicate $Q$ that is False 
    in the case of $\forall x \in S$:
    \begin{align*}
        &\boxed{P(a, b) : ``10 \mid (a \cdot b)", \text{ where } a \in S \text{ and } b \in T}\\
        &\boxed{Q(a) : ``3 \mid a," \text{ where } a \in S}
    \end{align*} 
    Thus $P$ is True for some but not all $y \in T$ (10 divides 20, but 10 
    does not divide 15), and $Q$ is False in the case of $\forall x \in S$ 
    (3 does not divide 2), and they are both not constant functions. (Statement 3)
    will now be $True \Rightarrow False$, which is False, and (Statement 4) 
    will be $False \Rightarrow True$, which is True. 
\end{enumerate}

\newpage
\q \textbf{Pierre Numbers}
\begin{enumerate} [label= (\alph*)]
    \item 
        $$PierreNumber(n) : \exists k \in \mathbb{Z}, \: n = 2^{2^k} + 1, 
        \text{ where } n \in \mathbb{N}$$
    \item $$\forall n \in \mathbb{N}, \: PierreNumber(n) \Rightarrow 
        (\exists k \in \mathbb{Z}, \: n = 2k + 1)$$
    \item $$\forall n \in \mathbb{N}, \: \exists p \in \mathbb{N},
        \: p > n \land PierreNumber (p)$$
    \item $$\forall n \in \mathbb{N}, \: \exists k \in \mathbb{Z}, \: n = 
        2^{2^k} + 1 \land k < 5 \Rightarrow Prime(n)$$
\end{enumerate}


\end{document}
