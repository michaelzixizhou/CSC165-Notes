\documentclass{article}
\usepackage{marginnote}
\usepackage[top=2.54cm, bottom=2.54cm, outer=8cm, inner=2.54cm, heightrounded, marginparwidth=4.46cm, marginparsep=1cm]{geometry}
\usepackage{lipsum}
\usepackage{parskip}
\usepackage{amsfonts}
\usepackage{marginfix} % floats sidenots
\usepackage{sidenotes}
\usepackage{amsmath}
\usepackage [english]{babel}
\usepackage [autostyle, english = american]{csquotes}
\MakeOuterQuote{"}
\usepackage{enumitem}
\title{Week 1: Mathematical Expression}
\author{Michael Zhou}
\date{2022-12-28}



\begin{document}
\maketitle

\section{Sets}

\setlength{\parindent}{0pt}
    

\textbf{Definiton 1.1.} Sets are a collection of distinct \textbf{elements}, it can
be finite or infinite. The size of a finite set \emph{A} is denoted by 
$\mid\emph{A}\mid$.
The \textbf{empty set} is denoted by $\emptyset$.

\marginnote{$^{1}$ It is important to note that the hierarchy of cardinalities is
$\#\mathbb{N} = \#\mathbb{Z} = \#\mathbb{Q} < \#\mathbb{R} = \#\mathbb{C}$.
Don't ask me why yet...} 
\textbf{Definiton 1.2.} The \textbf{cardinality} of a set is how many elements
are in the set.$^{1}$

\textbf{Examples}\\
Finite sets \hfill $\{a, b, c, d\}$, $\{2, 4, 10, 11\}$ \\
Set of tuples \hfill $\{(Ava Doe, \$ 700, 50), (Donald,\$ 670, 30\}$ 

Infinite sets \\ 
All natural numbers \hfill$\mathbb{N} = \{ 0, 1, 2, \ldots \} $ \\
All integers \hfill$\mathbb{Z} = \{\ldots, 2, -1, 0, 1, 2, \ldots\} $ \\
All positive integers \hfill$\mathbb{Z}^{+} = \{1, 2, \ldots\} $ \\
All rational numbers \hfill $ \mathbb{Q} $ \\
All real numbers \hfill $ \mathbb{R} $ \\
All complex numbers \hfill $ \mathbb{C} $ 

\marginnote{$^2$ Some definitions on binary operators and relations: \\
$=$ \hfill equals \\
$\in$ \hfill in \\ 
$\notin$ \hfill not in\\
$\subseteq$ \hfill subset \\
$\cap$ \hfill intersection \\
$\cup$ \hfill union \\ 
$\setminus$ \hfill set difference \\ 
$\times$ \hfill Cartesian product \\
$\mathcal{P}(\mathbb{A})$ \hfill power set}

A string of length 0 is called the \emph{empty string} and is denoted 
by $\epsilon$.

A set of all natural numbers greater or equal to five can be denoted by
$$ \{ x \mid x \in \mathbb{N} \; and \; x \geq 5 \} $$ 
The left of the $\mid$ describes the elements in the set in terms of \emph{x},
and the right part states the \emph{condition} on \emph{x}.

The set of all rational numbers can be denoted by
$$ \mathbb{Q} = \left\{ \frac{p}{q} \; \middle\vert \; p, q \in \mathbb{Z} \; and \; q \neq 0 \right\} $$

\section{Operations on sets} 
The size of a set, $\mid \emph{A} \mid$, is an example of a set operation. There
are many other common set operations.$^2$

\marginnote{Cartesian products are all \emph{pairs} $(a, b)$ where $a$ and $b$ are
elements of their respective sets.}

\textbf{Examples}\\
Returns booleans \hfill $x \in A$, $y \not\in A$, $A \subseteq B$, $A = B$ \\
Returns sets \hfill $A \cap B$, $A \cup B$, $A\setminus B$, $A \times B$, $\mathcal{P}(\mathbb{A})$

Power sets are sets containing \emph{all} subsets of $A$. If $A = {1, 2, 3}$, then
$$\mathcal{P}(\mathbb{A}) = \{\emptyset, \{1\}, \{2\}, \{3\}, \{1, 2\}, 
\{1, 3\}, \{2, 3\}, \{1, 2, 3\}\}$$

\newpage
\section{Functions}
\textbf{Definition 1.3.} Let $A$ and $B$ be sets. A \textbf{function} $ f
: A \to B $ is a mapping from elements in $A$ to elements in $B$. $A$ is the
\textbf{domain} of the function, and $B$ is the \textbf{codomain} of the 
function.

\textbf{Example} \\
Predecessor function \hfill $Pred: \mathbb{Z} \to \mathbb{Z} $ \\
defined by \hfill $Pred(x) = x - 1$ \\ 
$Pred(A) $ has the set  
$$ \{ \ldots, (-2, -3), (-1, -2), (0, -1), (1, 0), (2, 1), \ldots \} $$

This function would match each integer to the integer before it.

Functions can have multiple inputs.

\textbf{Example} \\
\marginnote{$^3$ k-ary refers to terms like unary, binary, ternary functions
that take one, two, and three inputs respectively.}
k-ary function$^3$ \hfill $f : A_1 \times A_2 \times \ldots \times A_k \to B$ \\
Addition operator (binary) \hfill $ + : \mathbb{R} \times \mathbb{R} \to \mathbb{R} $ \\
Predicate function \hfill $ f : \mathbb{N} \to $ \{True, False\}$^4$
\marginnote{$^4$ Can be represented as 1 and 0 respectively.}[2cm]

If an element $x$ is within the condition of the function, we say that $x$
\textbf{satisfies} $P$ when $P(x)$ is True.

\textbf{Definition 1.4.} A \textbf{predicate} function is defined by a codomain
of \{True, False\}. Predicates and sets are closely related.

\textbf{Example} \\
Set \hfill Predicate \\
$\{ x \mid x \in A \; and \:P(x) = True\}$. \hfill $ P : A \to$ \{True, False\} \\
$B \subseteq A$ \hfill $P : A \to $ \{True, False\} by $P(x) = $ True if $x \in A$ \\
\{0, 2, 4, \ldots \} \hfill  $ Even : \mathbb{N} \to $ \{True, False\}

\section{Summation and product notation}
\textbf{Definition 1.5.} The \textbf{summation notation} is used to express
sums of terms where each term follows a pattern.
$$ \sum_{i=1}^{100} \frac{i + i^2}{3 + i} $$
$i$ is the \emph{index of summation}, 1 and 100 are the \emph{lower} and 
\emph{upper bounds} of the summation.

\textbf{Definition 1.6.} The \textbf{product notation} is similar to the 
summation notation, but is used to abbreviate multiplication instead.
$$ \prod_{i=j}^{100} f(i) = f(j) \times f(j+1) \times \ldots \times f(k) $$

\marginnote{$^5$ These values are chosen so that the overall value of the 
expression is not changed when adding an empty summation or multiplying
by an empty product.}
The lower bound can be greater than its upper bound, in which case it
is an \emph{empty} summation or product$^5$ 
\begin{itemize}
    \item Summations have a sum of 0
    \item Products have a product of 1
\end{itemize}

\newpage
\section{Inequalities}
\marginnote{$^6$ These might seem obvious but it is important to express them mathematically.}
\textbf{Theorem 1.1.} For all real numbers $a$, $b$, and $c$, the following are true:$^{6}$
\begin{enumerate}[label=(\alph*)]
    \item If $a \leq b$ and $b \leq c$, then $a \leq c$.
    \item If $a \leq b$, then $a + c \leq b + c$.
    \item If $a \leq b$ and $c > 0$, then $ac \leq bc$.
    \item If $a \leq b$ and $c < 0$, then $ac \geq bc$.
    \item If $0 < a \leq b$, then $\frac{1}{a} \geq \frac{1}{b}$.
    \item If $a \leq b < 0 $, then $\frac{1}{a} \geq \frac{1}{b}$.
\end{enumerate}
\marginnote{$^7$ A strict equality is expressed by $>$ or $<$.}
If any of the above equalities are replaced with a strict equality, then the 
corresponding "then" equality will also be strict.$^7$

The implications of the inequalities is that adding or multipying by positive
numbers preserves inequalities, while multiplying by negative numbers or
taking reciprocals reverses inequalities. 

This distinguishes inequalities from equalities, since equailities always
preserve directionality. 

\textbf{Theorem 1.2.} For all non-negative real numbers $a$ and $b$, and
all strictly increasing functions $f : \mathbb{R}^{\geq 0} \to \mathbb{R}^{\geq 0}$,
if $a \leq b$, then $f(a) \leq f(b) $. This also follows the strict equality substitution: \\
if $a < b$, then $f(a) < f(b)$. 

This theorem implies that positive operations on strictly increasing functions
preserves equalities.

\section{Propositional logic}
\textbf{Definition 1.7.} A \textbf{proposition} is a statement that is either
True or False.

\textbf{Examples} 
\begin{itemize}
    \item $ 2 + 1 = 3$
    \item $ 4 + 3 < 8$
    \item Every even number greater than 2 is the sum of two prime numbers
    \item Python's \verb|list.sort| is correct on every input list
    
\end{itemize}

\textbf{Propositional variables} are used to represent propositions. The variable
names start at \emph{p} by convention. 

A \textbf{propositional/logical operator} is a predicate whose arguments 
must be either True or False.

A \textbf{propositional formula} is an expression built up from propositional
variables by applying the following operators.

\newpage
\section{Basic operators}
\marginnote{$^{8}$ There are two different \emph{or's}, the \emph{exclusive} and
the \emph{inclusive}. Exclusive is having one but not the other, inclusive
is having either or both. Usually, the inclusive \emph{or} is used. The exclusive
\emph{or} is denoted by $\oplus$.}
\begin{tabular}{|p{1cm} p{1cm} p{7cm}|}
    \textbf{NOT} & $\neg$ & Unary negation operator that flips the truth value. \\
    \textbf{AND} & $\land$ & Binary conjunction operator that returns True when both
    arguments are True. \\
    \textbf{OR} & $\lor$ & Binary disjunction operator that returns True when at least
    one argument is True. \\
\end{tabular}

These are some common operators on propositional variables.$^{8}$ 

\section{Implication operator}
\textbf{Definition 1.8.} The \textbf{implication} $p \implies q $ asserts 
that whenever $p$ is True, $q$ must be True. The statement $p$ is called
the \textbf{hypothesis} of the implication, and the statement $q$ is called
the \textbf{conclusion} of the implication.

\begin{center}
    
\begin{tabular}{|p{1cm} p{1cm} p{1cm}|}
    \hline
    $p$ & $q$ & $p \Rightarrow q$ \\
    \hline 
    False & False & True \\
    False & True & True \\
    True & False & False \\
    True & True & True \\
    \hline   
\end{tabular}
\end{center}
The two cases where $ p \implies q$ is True even though $p$ is False are
called \textbf{vacuous truth} cases. These are True because the statement
does not say anything about the behaviour of $q$ when $p$ is False. It simply
cannot be disproven when $p$ is False.

The following two formulas are equivalent to $p \implies q$:

$\neg p \lor q$ \hfill This uses the vacuous truth cases. \\
$\neg q \Rightarrow \neg p$ \hfill This is the \textbf{contrapositive} case.

\marginnote{$^{9}$ These cases might seem tricky, but it helps if you
compare the inputs with the truth table above.}
\textbf{Example$^{9}$} \\
If $p \implies q$ is \hfill "If you hate sweets, then you don't like candy". \\
$\neg p \lor q$ would be \hfill "You like sweets, or you don't like candy." \\
$\neg q \implies \neg p$ would be \hfill "You like candy, then you like sweets."

\textbf{Definition 1.9.} The \textbf{converse} of an implication switches
the hypothesis and conclusion. The converse of $p \implies q$ is $q \implies p$.
These two equations are not logically equivalent.

\section{Biconditional operator}
\marginnote{
\begin{tabular}{|p{0.8cm} p{0.8cm} p{1.5cm}|}
    \hline
    $p$ & $q$ & $p \iff q$ \\
    \hline 
    False & False & True \\
    False & True & False \\
    True & False & False \\
    True & True & True \\ 
    \hline   
\end{tabular}
}
\textbf{Definition 1.10.} The \textbf{biconditional} operator denoted by $p \iff q$ 
returns True when both $p \implies q$ and $q \implies p$ are True. In other words,
$p \iff q$ abbreviates $(p \Rightarrow q) \land (q \Rightarrow p)$.

This condition can be phrased any of the below: 
\begin{center}
"If $p$ then $q$, and if $q$ then $p$." \\
"$p$ if and only if $q$." \\
"$p$ iff $q$."
\end{center}

\textbf{Definition 1.11.} A \textbf{tautology} is a formula that is always
True for every possible assignment of values to its propositional variables. E.g. 
$(p \Rightarrow q) \Leftrightarrow (\neg p \lor q)$, $(\neg(p\lor q)) 
\Leftrightarrow (\neg p \land \neg q)$.

\newpage
\section{Predicate logic}
\marginnote{$^{10}$ You have already seen some real world examples of predicates: 
the operators = and $<$ both return True or False based on operands!}
So by now, you've heard of \textbf{predicates}, and how they have a codomain of True or False. 
We can extend this definition to be "A statement whose truth depends on one or
more variables from any set."$^10$

When we substitute values into a predicate, we obtain a proposition:
$P(x, y) $ is the statement $x^2 = y$ \hfill $P(5, 25)$ is True, $P(5, 24)$ is False.

We can complete the definiton of the above statements by giving the domain
of the prediate. In fact, \textbf{it is not a predicate if we do not give the domain.}

\begin{center}
$ P(x) : $ "$x$ is a power of 2," \quad where x$ \in \mathbb{N} $
\end{center}

\section{Quantification of variables}
Truth aggregation is when we want to express a predicate's truth values for
all elements of its domain, like the inequality $x^2 - 2x + 1 > 0$. 

There are two ways we can express this truth aggregation using quantifiers, 
which modify predicates by specifying how a variable should be interpreted.

\textbf{Definition 1.12.} The \textbf{existential quantifier}, $\exists$, 
abbreviates "there exists an element in the domain that satisfies the
given predicate." 

\marginnote{$^{11}$ These examples are stating that there must be \emph{at least}
one of the variable that satisfies the condition --- a continuous OR operation.}
\textbf{Examples$^{11}$}
\begin{center}
\begin{tabular}{|c p{7.54cm}|}
    $\exists x \in \mathbb{N}$, $x \geq 0$ & There exists a natural number
    $x$ that is greater than or equal to zero. \\
    $\exists y \in \mathbb{N}$, $y = 2^a$  & There exists a natural number 
    $y$ that is a power of 2.
\end{tabular}
\end{center}

\textbf{Definition 1.13.} The \textbf{universal quantifier}, $\forall$,
abbreviates "\emph{every} element in the domain satisfies the given predicate."

\marginnote{$^{12}$ In contrast to the above, this asks for \emph{all} 
elements to meet the condition, like a continuous AND operation.}
\textbf{Examples$^{12}$}
\begin{center}
\begin{tabular}{|c p{7.54cm}|}
    $\forall x \in \mathbb{N}$, $x \geq 0$ & Every natural number is
    $x$ that is greater than or equal to zero. \\
    $\forall y \in \mathbb{N}$, $y = 2^a$  & Every natural number is  
    $y$ that is a power of 2.
\end{tabular}
\end{center}

\section{Understanding multiple quantifiers}
It is important to note that the ordering of quantifiers \emph{do} matter in
some cases.

For commutative operators like addition and multiplication, order does not
matter. The universal operator is commutative if used consecutively:
$$\forall x \in S_1, \forall y \in S_2, P(x, y)$$
$$\forall y \in S_2, \forall x \in S_1, P(x, y)$$
These formulas are equivalent. In fact, we can combine these quantifications
since the variables have the same range!
$$\forall x, y \in S, P(x, y)$$
This reads, "every $x$ and $y$ in $S$ follows $P(x, y)$."

\newpage
The same follows for consecutive existential quantifiers:
$$\exists x \in S_1, \exists y \in S_2, P(x, y) \quad \quad 
\exists y \in S_2, \exists x \in S_1, P(x, y)$$
This can be summed up as $\exists x, y \in S, P(x, y)$, read as 
"there is at least one pair of elements $x$ and $y$ that satisfy $P(x,y)$."

This is \emph{not} the case for alternating quantifiers however.

\textbf{Example}
$$\forall a \in A, \exists b \in B, Likes(a, b)$$
"For every person $a$ in $A$, there exists a person $b$ in $B$, that $a$ likes $b$."
$$\exists b \in B, \forall a \in A, Likes(a, b)$$
"There exists a person $b$ in $B$, where for every person $a$ in $A$, $a$ likes $b$"

In both these cases, the first variable is \emph{independent} of the
second variable. Read nested quantifiers from left to right!

\section{Sentences in predicate logic}
With quantifiers, propositional operators, and predicates, we can represent 
statements using \textbf{sentences}.

\marginnote{$^{12}$ Sometimes a quantified variable will be refered to as 
a \textbf{bound} variable, and an unquantified variable a \textbf{free} 
variable.}
\textbf{Definiton 1.14.} A \textbf{sentence} is a formula with no unquantified 
\newline variable.$^{12}$ This ensures that the formula has a fixed truth value.

\textbf{Examples} \\
$\forall x \in \mathbb{N}, x^2 > y$ \hfill Not a sentence since $y$ is not bound. \\
$\forall x, y \in \mathbb{N}, x^2 > y$ \hfill Is a sentence since both variables are bound. 

\section{Manupulating negation}
For any formula, we can state its negation by preceding it by a $\neg$ symbol.

$\forall x \in \mathbb{N}, x \geq 0$ \hfill  $\neg(\forall x \in \mathbb{N}, x \geq 0)$

\marginnote{$^{13}$ Note that many of these rules lead into switching from 
\textit{and} to \textit{or}, and $\forall$ to $\exists$, and vice versa. Try 
not to memorize these, but to understand them.}
Though, sometimes it is hard to transliterate the formula. Instead, there are 
\textit{simplification rules}$^{13}$

$\neg(\neg p)$ \hfill p \\
$\neg(p \lor q)$ \hfill $(\neg p) \land \neg(q)$ \\
$\neg(p \land q)$ \hfill $(\neg p) \lor (\neg q)$ \\
$\neg(p \Rightarrow q)$ \hfill $p \land (\neg q) ^{14}$ \\
$\neg(p \Leftrightarrow q)$ \hfill $(p \land (\neg q)) \lor ((\neg p)\land q)$ \\
$\neg(\exists x \in S, P(x))$ \hfill $\forall x \in S, \neg P(x)$ \\
$\neg(\forall x \in S, P(x))$ \hfill $\exists x \in S, \neg P(x)$ 

\marginnote{$^{14}$ As a reminder, $p \Rightarrow q$ is equivalent to 
$ p \land \neg q$.} 

\section{Avoid commas}
Commas can cause ambiguity when connecting propositions.
$$P(x), Q(x)$$
Does this mean $P(x)$ \textit{and} $Q(x)$? Or $P(x)$ \textit{then} $Q(x)$? 

We must never use commas to \textbf{connect propositions}.

Commas have two valid uses only:
\begin{itemize}
    \item Immediately after variable quantification, or separating two variables 
        with the same quantification
    \item Separating arguments to a predicate
\end{itemize}

\textbf{Example} 
$$\forall x, y \in \mathbb{N}, \forall x \in \mathbb{R}, P(x, y) \Rightarrow Q(x, y, z)$$

\section{Defining predicates}
\textbf{Definiton 1.15.} Let $n, d \in \mathbb{Z}$. We want to say that $d$ 
divides $n$, or $n$ is divisible by $d$, when there exists a $k \in \mathbb{Z}$ 
such that $n = dk$. So, we will use the notation $d \mid n$ to represent 
"$d$ divides $n$." This is a \textit{binary} \textbf{divisibility predicate}. 

\textbf{Examples} \\
Let us express the statement "For every integer $x$, if $x$ divides 10, then 
it also divides 100" with the divisibility predicate, and without.

\textbf{Without the predicate}:
$$\forall x \in \mathbb{Z}, (\exists k \in \mathbb{Z}, 10 = kx) \Rightarrow 
(\exists k \in \mathbb{Z}, 100 = kx)$$
\marginnote{$^{15}$ Note that there are two different $k$ variables, we could 
also express this using $k_1$ and $k_2$.}
"For every integer $x$, if there exists an integer $k$ such that $kx = 10$, then 
for another integer $k$, $kx = 100$."$^{15}$

\textbf{With the predicate}:
$$\forall x \in \mathbb{Z}, x \mid 10 \Rightarrow x \mid 100$$ 
Much easier, isn't it? 

We can use this definiton to formally define prime numbers.

\textbf{Definition 1.16.} Let $p \in \mathbb{N}$. A \textbf{prime} number is greater 
than 1, and the only natural numbers that divide it are 1 and itself. Primes are
restricted to being positive. 

\textbf{Example} \\
Let $Prime(p)$ denote that "$p$ is a prime number."
$$Prime(p) : p > 1 \land (\forall d \in \mathbb{N}, d \mid p \Rightarrow 
d = 1 \lor d = p), \quad p \in \mathbb{N}$$
Let us express the property that "there are infinitely many primes." \\
How do we express \textit{infinitely many}? Since we know that $ \mathbb{N}$ is
infinite, we can express the statement as "every natural number has a prime 
number larger than it."
$$\forall n \in \mathbb{N}, \exists p \in \mathbb{N}, p > n \land Prime(p)$$

\marginnote{$^{16}$ First conjectured by Pierre de Fermat in 1637, he states that 
the margins of the text \textit{Arithmetica} were too narrow to fit his proof!}
\textbf{Definition 1.17.} \textbf{Fermat's Last Theorem} states that there are 
no three positive integers $a$ , $b$, and $c$ that satisfy $a^n + b^n = c^n$ for 
any integer $n > 2$.$^{16}$

Let us express this theorem using predicate logic. 

\newpage
Which of these variables are quantified? $n$ is certainly bound to the range of 
for all $n > 2$ and $n$ being an integer. $a$, $b$, and $c$ are not specifically 
bound, but since the theorm states \textit{None} of them satisfy the statement, 
we can say "there does not exist" instead.
$$\forall n \in \mathbb{N}, n > 2 \Rightarrow \neg (\exists a, b, c \in 
\mathbb{Z}^+, a^n + b^n = c^n )$$
Following negation rules, we can push this negative inwards closer to the predicates.
$$\forall n \in \mathbb{N}, n > 2 \Rightarrow (\exists a, b, c \in 
\mathbb{Z}^+, a^n + b^n \ne c^n)$$

\section{Formula conventions}
Operation precedence in decreasing order:
\begin{center}
\begin{enumerate}
    \item $\neg$
    \item $\lor, \land$
    \item $\Rightarrow, \Leftarrow$
    \item $\forall, \exists$
\end{enumerate}
\end{center}
Combinations of operations at the same level \textit{must} be disambiguated 
using parentheses. 

The $\lor$ and $\land$ operators are \textit{associative}, meaning that their 
orders do not matter. But the implicator operator is \textit{not associative}. 

Variable naming conventions state that variables should have distinct names 
within the same formula.
$$(\forall x \in \mathbb{N}, f(x) \geq 5) \lor (\exists x \in \mathbb{N}, f(x) < 5)$$
Although the above is correct since variables only exist in the scope of their 
parentheses, we still prefer to use the following:
$$(\forall x \in \mathbb{N}, f(x) \geq 5) \lor (\exists y \in \mathbb{N}, f(y) < 5)$$

\end{document}
