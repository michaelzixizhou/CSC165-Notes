\documentclass{article}
\usepackage[top=2.54cm, bottom=2.54cm, outer=8cm, inner=2.54cm, heightrounded, marginparwidth=4.46cm, marginparsep=1cm]{geometry}
\usepackage{lipsum}
\usepackage{parskip}
\usepackage{amsfonts}
\usepackage{marginfix} % floats sidenotes
\usepackage{sidenotes}
\usepackage{amsmath}
\usepackage [english]{babel}
\usepackage [autostyle, english = american]{csquotes}
\MakeOuterQuote{"}
\usepackage{enumitem}
\usepackage{amssymb}

\newcounter{defcount}
\newcounter{excount}
\newcounter{thcount}
\newcommand\df{\stepcounter{defcount} \textbf{Definition 2.\thedefcount}. }
\newcommand\ex{\stepcounter{excount} \textbf{Example 2.\theexcount}. }
\newcommand\tr{\stepcounter{thcount} \textbf{Theorem 2.\thethcount}. }

\title{Week 2: Introduction to Proofs}
\author{Michael Zhou}
\date{2023-01-01}

\begin{document}
\maketitle

\begin{itemize}
    \item How can we figure out whether a statement is True or False?
    \item If we do know that a statement is True, how can we convince others?
    \item How can we read someone else's proof?
\end{itemize}

\section{Examples of proofs}
\df A \textbf{mathematical proof} is how we communicate ideas about the truths 
of a statement to others. 

The following examples will be divided into three or four parts:
\begin{enumerate}
    \item The statement we are proving or disproving. 
    \item A translation of the statement into predicate logic, describes the 
        logical structure we will be using.
    \item A discussion to try to gain some intuition about why the statement is True. 
        These are written informally, and often the hardest part about developing a proof.
    \item The formal proof, the standalone final product of the earlier work.
\end{enumerate}

\ex Prove that $15 \cdot 3^2 - 7 = 7 + (19+3)^2 / 4$.

\textit{Translation.} Since this statement has no logical operators, the 
translation is simply itself.

\textit{Discussion.} You can check if this is True by entering both sides into 
a calculator. 

\textit{Proof.} This statement is True because both sides equal 128. 

\ex Prove that there exists a power of two bigger than 1000.

\textit{Translation.} "There exists" translates to an existential quantifier, 
and "powers of 2" have the form $2^n$, where $n$ is a natural number.
$$\exists n \in \mathbb{N}, 2^n > 1000$$
\textit{Discussion.} This must be True since you know that powers of 2 can 
grow to infinity, so you just need to do some calculations for a large value 
of $n$.

\textit{Proof.} Let $n$ = 10. \\
Then $2^ {10}$ = 1024, which is greater than 1000.\sidenote{This is easily 
checkable with a calculator, so no need for further verification.}

Drawing from this example, in general, we only have to prove for one element in 
existentence proofs. 

\newpage
\ex Prove that every real number $n$ greater than 20 satisfies the inequality 
$1.5n - 4 \geq 3$. 

\textit{Translation.} "Every real number $n$" translates to a universal 
quantifier, and we can restrict $n$ to be greater than 20 as a condition on 
the hypothesis. 
$$\forall n \in \mathbb{R}, n > 20 \Rightarrow 1.5n - 4 \geq 3$$
\textit{Discussion.} You could gain intuition by substituting numbers for 
$n$ that are greater than 20, but that does not serve to prove \textit{all} 
real numbers. Instead, you could perform some algebraic manipulation on the 
formula to preserve the inequality $n > 20$: multiplying by 1.5, and 
subtracting 4. 

\textit{Proof.} Let $n \in \mathbb{R}$ be an \textit{arbitrary} real number. 
Assume that $n > 20$. We want to prove that $1.5n - 4 \geq 3$.\sidenote{We could 
also just write "Let $n \in \mathbb{R}$" instead.}
\begin{align*}
    n &> 20\\
    1.5n &> 30\\
    1.5n - 4 &> 26\\
    1.5n - 4 &\geq 3
\end{align*}

We can observe some standard patterns that these proofs always follow:

\begin{center}
\begin{tabular}{|l l l|}
    Existential & $\exists x \in S, P(x).$ & Let $x = a$, proof that $P(x)$ is True.\\ 
    Universal & $\forall x \in S, P(x).$ & Let $x \in S$, proof that $P(x)$ is True.\\
    Implication & Assume $p$ & Proof that $q$ is True. 
\end{tabular}
\end{center}

\section{What goes into a proof?}
\subsection{Proof header: setting up the proof}
\df Every proof should start with a \textbf{proof header}. The main purpose 
of a proof header is to introduce all variables and assumptions used in the 
proof. Pay attention to ordering, as we have learned about alternating 
quantifiers.

\begin{center}
\begin{tabular}{|l l|}
    Existential & "Let $x = 5$."\\
    Universal & "Let $x \in S$." or "Let $x$ be an arbitrary element of S."\\
    Local & "Let $\epsilon = x - x^2$." 
\end{tabular}
\end{center}

\ex $\forall x \in \mathbb{N}, P(x) \Rightarrow Q(x).$ 

\begin{center}
\fbox{Let $x \in \mathbb{N}$. Assume $P(x)$.}
\end{center}

\ex $\forall x \in \mathbb{N}, P_1(x) \land P_2(x) \land P_3(x).$ 

\begin{center}
\fbox{Let $x \in \mathbb{N.}$ Assume that $P_1(x), P_2(x)$, and $P_3(x)$ are all True.} 
\end{center}

\ex $P(x): "x^3 < 10x + 300",$ where $x \in \mathbb{N}$. Suppose we are 
proving a statement in the form of $\forall x \in \mathbb{N}, P(x) \Rightarrow Q(x)$. 

\begin{center}
\fbox{Let $x \in \mathbb{N}.$ Assume that $P(x)$ is True.} 
\end{center}

\ex $\forall x \in \mathbb{R}, \forall y \in \mathbb{N}, x > 10 \land y < z 
\Rightarrow \exists z \in \mathbb{R}, P(x,y,z)$. Let us write the proof header 
for this statement.

\textit{Proof.} Let $x \in \mathbb{R}$ and let $y \in \mathbb{N}$. Assume that $x > 10$ 
and that $y < x$. Let $z = \_\_\_\_\_ .$ We will prove that $P(x,y,z)$ is True.
\sidenote{Note that we picked apart the complex statement and focused on the 
core --- the predicate $P(x,y,z)$.}

$[$\textit{Proof body goes here.}$]$

\newpage
\subsection{Proof body: the chain of reasoning}
\df The \textbf{proof body} contains the actual reasoning showing that a 
statement must be True.

\df Proof bodies consist of a seqeunce of True statements called \textbf{deductions}, 
where each statement logically followings from a combination of the following sources 
of truth: 
\begin{itemize}
    \item Definitions
    \item Assumptions (made in proof header)
    \item Previous deductions (made early in proof body)
    \item External true statements
\end{itemize}

The proof body is a \textit{chain} in the sense that it uses statements already 
known to be True, then makes deductions until finally proving the statement --- 
A proof is over when a deduction that is the statement being proven is written. 

\ex A proof body is usually consisted of two parts, the deduction, and the reason 
for the deduction. 
\begin{itemize}
    \item "Since (reason), (deduction)."
    \item "Because we know (reason), we can conclude (deduction)."
    \item "It follows from (reason) that (deduction) is also True/holds."
\end{itemize}

\subsection{Logical deductions}
\df The most common form of logical deduction in proofs is \textbf{modus 
ponens}.\sidenote{Also known as implication elimination. } 
This rule says that if we know $p$ and $p \Rightarrow q$ are both 
True, then we can conclude that $q$ is also True. This matches the 
intuition of implication.

\ex Because we know $x > 10$ and that $x > 10$ implies $x^2 - x > 90$, 
we can conclude that $x^2 - x > 90$.

\df Another common form of logical deduction is \textbf{universal 
instantiation}. This rule says that if we already know a universal like 
$\forall x \in S, P(x)$, and a variable $y$ whose value is an element in 
the domain $S$, then we can conclude that $P(y)$ must be True. This matches 
the intuition for universally-quantified statements.

\ex Because we know that $y \in \mathbb{N}$ and that $\forall x \in \mathbb{N}, 
x^2 + 5x + 4$ is not prime, we can conclude that $y^2 + 5y + 4$ is not prime. 

\subsection{Writing reasons and deductions}
You must provide explicit reasons for all statements in a proof. 

However, there are two exceptions:
\begin{itemize}
    \item Any deduction verifiable by a calculator, like $100 < 3 \cdot 4$.
    \item Any basic manipulation of an equality or inequality, like $x > 4$ 
        to $2x > 8$.
\end{itemize}

For other types of reasonings --- definitions, assumptions, prior deductions, 
and external facts --- make references explicitly.

\newpage
\subsection{The direction of a proof}
For a normal arithmetic calculation, we would want to simplify a complex 
statement into its simplest form. But for a proof, we want to start from the 
simple, known statement, and derive our target deduction from them. 

We must always write our proofs in a \textbf{top-down} order to disambiguate 
statements. Think of it as reading an essay, but in the context of numbers. 

\section{A new domain: number theory}
Mathematical domains can specify common definitions and properties that 
we can use freely in proofs. The first domain we will be exploring is 
number theory, and formalizing what we know about numbers. We will start off 
with the divisibility operator from the previous chapter. 

\df Let $n, d \in \mathbb{Z}$. We say that $d$ divides $n$, or $n$ is 
divisible by $d$, if and only if there exists a $k \in \mathbb{Z}$ such 
that $n = dk$. We can use the notation $d \mid n$ to represent "$d$ divides 
$n$," and call $d$ a divisor of $n$, and $n$ a multiple of $d$.

\ex Prove that $23 \mid 115$.

\textit{Translation.} We can expand on the definition of divisibility to 
rewrite this statement:
$$\exists x \in \mathbb{Z}, 115 = 23x$$

\textit{Discussion.} We can just divide 115 by 23.

\textit{Proof.} Let $x = 5$. Then $115 = 23 \cdot x = 23 \cdot 5$.

\ex Prove that there exists and integer that divides 104.

\textit{Translation.} "There exists" translates into an existential quantifier, 
and "an integer" translates into the set of integers. We can use the 
divisibility operator to represent "divides 104", and expand on that using 
the definition of the divisibility operator.\sidenote{Since the variables share the same set, we can group them together.}

$$\exists n \in \mathbb{Z}, n \mid 104$$
$$\exists a, n \in \mathbb{Z}, 104 = an$$


\textit{Discussion.} We only need to find one example that satisfies this 
statement. Since 104 happens to be an even number, we know that it will be 
divisible by 2 at the very least.

\textit{Proof.} Let $n = 2$ and let $a = 52$. Then $104 = an$.

\section{Alternating quantifiers revisited}
Recall from the previous chapter that we said changing the order of 
an existential and universal quantifier changed the meaning of a statement. 
Now, we will investigate this affects variable introductions in a proof.

\newpage
\ex Prove that all integers are divisible by 1. 

\textit{Translation.} "All integers" translates into a universal quantifier, 
and proving that it is divisible by 1 uses an existential quantifier through 
the definition of divisibility. 
$$\forall n \in \mathbb{Z}, \exists k \in \mathbb{Z}, n = 1 \cdot k$$

\textit{Discussion.} We can prove for $n = k$, but how do we introduce the 
variables? We must follow the same order they are quantified in the 
statement.

\textit{Proof.} Let $n \in \mathbb{Z}$, and let $k = n$. Then $n = 1
\cdot n = 1 \cdot k$.

\section{False statements and disproofs}
\df A \textbf{disproof} is a proof that the negation of the statement is True. 
In other words, if $\neg X$ is True, then $X$ must be False.

\ex Disprove "every natural number divides 360."

\textit{Translation.} "Every natural number" translates to the set of 
natural numbers and a universal quantifier. We have to disprove this statement 
though, so we will use the converse.\sidenote{We will use the simplication rules 
for negation from the previous chapter to find the converse.}
$$\neg(\forall n \in \mathbb{N}, n \mid 360)$$
$$\exists n \in \mathbb{N}, n \nmid 360$$

\textit{Discussion.} We can find a natural number that does not divide 
360. 

\textit{Proof.} Let $n = 7$. Then $n \nmid 360$, since $\frac{360}{7}$ is not an integer.

\ex Disprove "for all natural numbers $a$ and $b$, there exists a natural 
number $c$ which is less than $a+b$, and greater than both $a$ and $b$, such 
that $c$ is divisible by $a$ or by $b$."

\textit{Translation.} This is a fairly complex statement but we can break it down: 
"for all natural numbers" translates to a universally quantified variable in 
the set of natural numbers; "less than" and "greater than" translate to inequalities 
in the form of $a \geq b \geq c$; "is divisible by" translates to the divisibily 
operator. 
$$\forall a, b \in \mathbb{N}, \exists c \in \mathbb{N}, c < a + b \land 
c > a \land c > b \land (b \mid c \lor a \mid c)$$

We can derive the negation now:
\begin{center}
$\neg(\forall a, b \in \mathbb{N}, \exists c \in \mathbb{N}, c < a + b \land 
c > a \land c > b \land (b \mid c \lor a \mid c))$ \\
$\exists a, b \in \mathbb{N}, \neg (\exists c \in \mathbb{N}, c < a + b \land 
c > a \land c > b \land (b \mid c \lor a \mid c))$ \\
$\exists a, b \in \mathbb{N}, \forall c \in \mathbb{N}, \neg (c < a + b \land
c > a \land c > b \land (b \mid c \lor a \mid c))$ \\
$\exists a, b \in \mathbb{N}, \forall c \in \mathbb{N}, c \geq a + b \lor 
c \leq a \lor c \leq b \lor (\neg (a \mid c \lor b \mid c))$

$\exists a, b \in \mathbb{N}, \forall c \in \mathbb{N}, c \geq a + b \lor 
c \leq a \lor c \leq b \lor a \nmid c \land b \nmid c))$ \\
\end{center}

\textit{Discussion.} "There exists natural numbers $a, b$ such that for every 
natural number $c$, $c \geq a + b$ or $c \leq a$ or $c \leq b$ or neither 
$a$ or $b$ divide $c$." We will pick values for $a$ and $b$ since there is 
an existential quantifier. It is helpful to pick small values so that we do not 
have to worry about the numbers in between. 

\textit{Proof.} Let $a = 2 $ and $b = 2$, and let $c \in \mathbb{N}$. We 
now need to prove that 
$$c \geq a + b \lor c \leq a \lor c \leq b \lor (a \nmid c \land b \nmid c).$$

Substitute the values for $a$ and $b$, and we have
$$c \geq 4 \lor c \leq 2 \lor 2 \nmid c.$$

To prove an OR, we only need one of the three parts to be True. But, which 
part is True depends on the value of $c$. For example, not all numbers $c$ are 
greater or equal to 4. So we can split up the proof into three cases for different 
$c$ values: numbers $\geq 4$, numbers $\leq 2$, and the single value 3. 

\textbf{Case 1.} Assume $c \geq 4$, then the first part is True.

\textbf{Case 2.} Assume $c \leq 2$, then the second part is True.

\textbf{Case 3.} Assume $c = 3$, then the third part is True since $2 \nmid 3$.

Since all possible cases result in True, we can conclude that the statement is always True.

\section{Proof by cases}
The above proof illustrated a new technique. 

\df A \textbf{proof by cases} divides a domain into different parts, and writes 
a separate argument for each part. Formally, we pick a set of unary predicates 
such that for every element $x$ in the domain, $x$ satisfies at least one of 
the predicates --- thus, these predicates are $exhaustive$.\sidenote{Since 
a set and a predicate have an equivalence, we can also portray this as 
dividing the domain into subsets, and proving the statement for each subset.}

In other words, a proof by cases combines simpler proofs to form a whole proof. 
We introduce the following theorem as a natural use of proof by cases in number theory.

\tr The \textbf{Quotient-Remainder Theorem} states that for all $n \in \mathbb{Z}$ 
and $d \in \mathbb{Z}^+$, there exists $q, r \in \mathbb{Z}$ such that 
$n = qd + r$ and $0 \leq r < d$. Moreover, these $q$ and $r$ are \textit{unique}. 

\df Let $n, d, q, r$ be the variables from the above theorem. We say that 
$q$ and $r$ are the \textbf{quotient} and \textbf{remainder}, respectively, 
when $n$ is divided by $d$.

This theorem tells us that for any divisor $d \in \mathbb{Z}^+$, we can separate 
all possible integers into $d$ different groups, corresponding to their 
possible remainders --- between 0 and $d-1$ --- when divided by $d$. 

\ex Prove that for all integers $x, \: 2 \mid x^2 + 3x$.

\textit{Translation.} Using the divisibility predicate or expanding it: 
$$\forall x \in \mathbb{Z}, 2 \mid x^2 + 3x$$
$$\forall x \in \mathbb{Z}, \exists k \in \mathbb{Z}, x^2 + 3x = 2k$$

\textit{Discussion.} We want to factor out a 2 from $x^2 + 3x$, but this only 
works if $x$ is even. If $x$ is odd, then both $x^2$ and $3x$ will be odd, 
therefore producing an even number. We can use the Quotient-Remainder Theorem to 
formalize this thought.

\textit{Proof.} Let $x \in \mathbb{Z}$. By the QRT, we know that when $x$ is 
divided by 2, there will either be a remainder of 1 or 0. So, we can divide the 
proof into two cases. 

\textbf{Case 1.} Assume the remainder of $x/2$ is 0. In other words, $\exists q 
\in \mathbb{Z}, x = 2q + 0$. Let $k = 2q^2 + 3q$, we will show that $x^2 + 3x = 2k$.
\begin{align*}
    x^2 + 3x &= (2q)^2 + 3(2q) \\
             &= 4q^2 + 6q \\
             &= 2(2q^2 + 3q) \\
             &= 2k
\end{align*}

\textbf{Case 2.} Assume the remainder of $x/2$ is 1. In other words, $\exists q 
\in \mathbb{Z}, x = 2q + 1$. Let $k = 4q^2 + 10q + 4$, we will show that 
$x^2 + 3x = 2k$.\sidenote{As a reminder, we are using binomial expansion.}
\begin{align*}
    x^2 + 3x &= (2q + 1)^2 + 3(2q + 1)\\
             &= 4q^2 + 4q + 1 + 6q + 3 \\
             &= 4q^2 + 10q + 4 \\
             &= 2(2q^2 + 5q + 2) \\
             &= 2k 
\end{align*}

\section{Generalizing statements}
We will investigat how to generalize existing knowledge into more generic 
and powerful forms. Let's give an example \textit{without} generalization.

\ex Prove that for all integers $x$, if $x$ divides $(x + 5)$, then $x$ also 
divides 5. 

\textit{Translation.} There is a universal quantifier and an implication 
in this statement. 
$$\forall x \in \mathbb{Z}, x \mid (x+5) \Rightarrow x \mid 5$$

\textit{Discussion.} You can assume $x$ divides $x + 5$, and prove that 
$x$ also divides 5. Need to turn $x+5 = k_1x$ into $5 = k_2x$. We can use the 
definition of divisibility. We will probably substitute $k_1$ into $k_2x$ to prove 
$k_2x = 5$.

\textit{Proof.} Let $x \in \mathbb{Z}$. Assume that $x \mid (x+5)$, such that 
$\exists k_1 \in \mathbb{Z}, x+5 = k_1x$. We want to prove that $\exists k_2 \in 
\mathbb{Z}, 5 = k_2x$. Let $k_2 = k_1 - 1$.\sidenote{Our intuition for choosing 
$k_2 = k_1 -1$ is that we want to get rid of the $x$ when we substitute 
$(x+5)$ for $k_1x$, leaving the singular 5 to prove our statement.}
\begin{align*}
    k_2x &= (k_1 - 1)x \\
         &= k_1x - x \\
         &= (x + 5) - x \\
         &= 5
\end{align*}
What did we do in this proof? 
\begin{enumerate}
    \item To prove an implication, we assume the hypothesis to be True in order 
        to prove the conclusion. We are \textit{not} proving the hypothesis.
    \item We used the definition of divisibility to include a variable $k_1 
        \in \mathbb{Z}$, and the same for $k_2$.
    \item Finally, we proved that if a factor $k_1$ exists for $x$ when it divides 
        $(x+5)$, another factor $k_1$ \textit{also} exists for $x \mid 5$.
\end{enumerate}

\newpage 
\subsection{Generalizing our example}
\ex The meta-technique \textbf{generalization} takes in a True statement 
with its proof, and replaces a concrete value in the statement with a 
universally quantified variable.

Consider the statement from the previous example, $\forall x \in \mathbb{Z}, 
x \mid (x + 5) \Rightarrow x \mid 5$. It seems like the "5" does not serve 
any special purpose --- we can probably replace it with another variable. 

Rather than replacing 5 with another number, we can replace it with a 
universally quantified variable, and proving the corresponding statement. We 
will know that we can replace the "5" with \textit{any} natural number and 
the statement will still hold. 

\ex Prove that for all $d \in \mathbb{Z}$, and for all $x \in \mathbb{Z}$, 
if $x$ divides $(x+d)$, then $x$ also divides $d$. 

\textit{Translation.} The translation follows our previous example:
$$\forall d, x \in \mathbb{Z}, \Bigl( (\exists k_1 \in \mathbb{Z}), 
x+d = k_1x \Rightarrow (\exists k_2 \in \mathbb{Z}, d = k_2x)\Bigr)$$

\textit{Discussion.} We should be able to use the same sets of calculations.

\textit{Proof.} Let $d, x \in \mathbb{N}$. Assume that $x \mid (x+d)$, there 
exists $k_1 \in \mathbb{Z}$, such that $x + d = k_1x$. We want to prove 
that there exists $k_2 \in \mathbb{Z}$ such that $d = k_2x$. Let $k_2 = k_1 - 1$.
\begin{align*}
    k_2x &= (k_1 -1)x \\
         &= k_1 -x \\
         &= (x+d) - x \\
         &= d
\end{align*}

Unlike the the previous proof, we \textbf{generalized} the proof so that it 
does not depend on the value of 5. By doing this, we have gone from one 
statement being True, to an infinite number of statements being True. 

\section{Characterizations}
We will now present two related examples that both show how to prove a 
biconditional. We will also find alternative useful characterizations of 
definitions. In particular, we will show that prime numbers are exactly the 
numbers greater than 1 that satisfy the following predicate: 
$$Atomic(n): \forall a, b \in \mathbb{N}, n \nmid a \land
n \nmid b \Rightarrow n \nmid ab, \quad \text{where } n \in \mathbb{N}$$

\ex We will prove the following statement.\sidenote{Or said in English, "every
    number that is greater than one and atomic must also be prime."}
$$\forall n \in \mathbb{N}, \Bigl( n > 1 \land (\forall a, b \in \mathbb{N}, 
n \nmid a \land n \nmid b \Rightarrow n \nmid ab) \Bigr) \Rightarrow Prime(n)$$
It's not immediately clear how we should use the hypothesis to prove the 
conclusion. So, we can try writing the \textit{contrapositive} of the statement. 
$$\forall n \in \mathbb{N}, \neg Prime(n) \Rightarrow \Bigl( n \leq 1 
    \lor (\exists a, b \in \mathbb{N}, n \nmid a \land n \nmid b \land 
    n \mid ab)$$
Now, we can assume that $n$ is \textit{not} prime, and we only need an 
existential proof or $n \leq 1$. We can prove the contrapositive as it will 
also be a proof for the original.

\newpage
\textit{Discussion.} We are going to assume that $n$ is not prime, and it's
greater than 1. First, let's look at the definition of $Prime$ and negate it. 
$$Prime(n) : n > 1 \land (\forall d \in \mathbb{N}, d \mid n \Rightarrow 
d = 1 \lor d = n)$$
$$\neg Prime(n) : n \leq 1 \lor (\exists d \in \mathbb{N}, d \mid n \land d 
\neq 1 \land d \neq n)$$
So if we assume that $n > 1$, then we can also assume that there exists 
a number $d$ that divides $n$ that is not 1 or $n$.

\textit{Proof.} Let $n \in \mathbb{N}$. Assume that $n$ is not prime. Then 
by negating the definition of prime, either $n \leq 1$ or there exists 
$d \in \mathbb{N}$, $d \mid n \land d \neq 1 \land d \neq n$. We divide 
our proof into two cases based on the two parts.

\textbf{Case 1.} Assume $n \leq 1$. Then the first part is True. 

\textbf{Case 2.} Assume $\exists d \in \mathbb{N}, d \mid n \land d 
\neq 1 \land d \neq n$.

Expanding on the divisibility predicate, this means that there exists \linebreak 
$k \in \mathbb{N}$ such that $n = dk$. We will prove the second part of the 
OR, listed below. Let $a = d$ and $b = k$. We want to prove that $n \nmid a$, $n\nmid b$ 
and $n \nmid ab.$ 
$$\exists a, b \in \mathbb{N}, n \nmid a \land n \nmid b \land n \mid ab$$
We will substitute $d$ and $k$ into the statement:
\begin{align}
    n \nmid a\land n \nmid b \land n \nmid ab &= n \nmid d \land n \nmid k 
    \land n \mid dk \\
                                             &= dk \nmid d \land dk \nmid k
                                             \land dk \mid dk
\end{align}
All cases of (2) are True: 
\begin{enumerate}
    \item $dk \nmid d$ is always True since $d \neq n$ and $n = dk$, meaning 
        $k$ cannot equal to 1. 
    \item $dk \nmid k$ is always True since $d \neq 1$.
    \item $dk \mid dk$ is always True since a number $dk$ is always a factor 
        of itself. 
\end{enumerate}
What we just did was prove that if $n$ is greater than 1, and satisfies 
the $Atomic$ predicate, then it must be prime. But what about $n = 5$? It 
does not tell us whether 5 satisfies the property since a prime number would 
have a False hypothesis, leaving an unknown conclusion.\sidenote{Recall the
\textbf{vacuous truth} cases from the previous chapter for implication 
operators.} Next, we will prove the \textit{converse} of the implication.

\ex Prove the following converse of example 2.20.\sidenote{Or said in English, "
Every number that is prime must also be greater than one and atomic."}
$$\forall n \in \mathbb{N}, Prime(n) \Rightarrow \Bigl( n > 1 \land (
    \forall a, b \in \mathbb{N}, n \nmid a \land n \nmid b \Rightarrow n 
    \nmid ab) \Bigr) $$
\textit{Discussion.} We can try an example to gain some intuition. Consider 
the prime $n = 7$ and arbitrary numbers $a$ and $b$. When both $a$ and $b$ 
do not have 7 as a divisor, like $a = 12$ and $b = 10$, $a \cdot b =120$ 
also does not have 7 as a divisor (satisfying $Atomic(n)$). 

But how do we prove this? We could write out a prime factorization of 
$a$ and $b$: $a \cdot b = 2 \cdot 2 \cdot 3 \cdot 2 \cdot 5$. This clearly 
shows that 7 is not a divisor of $a \cdot b$.

The problem with this proof is that we would have to prove that every 
number has a unique prime factorization, which is quite unnecessary. 
Instead, we will use the following two facts that are easier to prove.\sidenote{
These rely on properties of the \textit{greatest common factor}, which will 
be discussed in the next section.}

\newpage 
\begin{align*}
    \forall n, m \in \mathbb{N}, Prime(n) \land n \nmid m \Rightarrow 
    (\exists r, s \in \mathbb{Z}, rn + sm = 1)\tag*{(Claim 1)} \\
    \forall n, m \in \mathbb{N}, Prime(n) \land (\exists r, s \in \mathbb{Z}, 
    rn + sm = 1) \Rightarrow n \nmid m \tag*{(Claim 2)}
\end{align*}
How might we setup a proof using these claims? Since we are assuming $n$ to 
be prime, we can have two numbers $a, b$ that are not divisible by $n$. Using 
Claim 1 twice, there exists $r_1, s_1$ (for $a$) and $r_2, s_2$ (for $b$) 
such that:
\begin{align*}
    r_1n + s_1a &= 1 \\
    r_2n + s_2b &= 1
\end{align*}
We want to conclude that $ab$ is also not divisible by $n$, so we will use 
Claim 2, which says "to conclude that $ab$ is not divisible by $n$, it 
suffices to find $r, s$ such that $rn + s(ab) = 1$." We can then find $r, s$ 
by multiplying the two equations together:
$$r_1r_2n^2 + r_2s_1an + r_2s_2bn + s_1s_2ab = 1$$
This can be rewritten as 
$$(r_1r_2n + r_2s_1a + r_1s_2b)n + (s_1s_2) (ab) = 1$$

\textit{Proof.} Let $n \in \mathbb{N}$. Assume that $n$ is prime. We need to 
prove that $n > 1$ and $Atomic(n)$ are True. 

\textbf{Case 1.} $n > 1$. The definition of prime tells us that this is True.

\textbf{Case 2.} $Atomic(n)$ is True. 

We want to prove that ($\forall a, b \in \mathbb{N}, n \nmid a \land
n \nmid b \Rightarrow n \nmid ab, \: \text{where } n \in \mathbb{N}$). 
Let $a, b \in \mathbb{N}$, and assume that $n \nmid a$ and $n \nmid b$. We 
want to prove that $n \nmid ab$.\sidenote{Just like all the other 
implication proofs, you should be used to this pattern by now.}

We will first prove that there exists $r_3, s_3 \in \mathbb{N}, r_3n + 
s_3ab = 1$. By Claim 1 and the assumption that $n$ is prime, there exists 
$r_1, s_1, r_2, s_2 \in \mathbb{Z}$ such that $r_1n + s_1a = 1$ and 
$r_2n + s_2b = 1$. Let $r_1r_2n + r_2s_1a + r_1s_2b = r_3$, and $s_1s_2 = s_3$.
\begin{align*}
    (r_1n + s_1a)(r_2n + s_2a) &= 1 \\
    (r_1r_2n + r_2s_1a + r_1s_2b)n + (s_1s_2) (ab) &= 1 \\ 
    r_3n + s_3ab &= 1
\end{align*}
By Claim 2, we have now proved the statement that $n \nmid ab$. In other words, 
we have proved both cases, thus concluding our proof. 

\subsection{A summary}
We have proven both the following statements:\sidenote{Recall how a biconditional 
operator, $\Leftrightarrow$, is defined by implications from both directions being True.}
\begin{align*}
    \forall n \in \mathbb{N}, n > 1 \land Atomic(n) \Rightarrow Prime(n) \\
    \forall n \in \mathbb{N}, Prime(n) \Rightarrow n > 1 \land Atomic(n) \\
    \forall n \in \mathbb{N}, n > 1 \land Atomic(n) \Leftrightarrow Prime(n)
\end{align*}
Thus, we have given a \textit{characterization} or \textit{alternate definition} 
of prime numbers. Equivalent characterizations are useful as they give a 
different way to look at the same concept. 

\section{Greatest common divisor}
\df Let $m, n$ be natural numbers which are not both 0. The \textbf{greatest 
common divisor (gcd)} of $m$ and $n$, denoted gcd$(m,n)$, is the highest 
natural number $d$ such that $d$ divides both $m$ and $n$. 

We define gcd$(0, 0)$ to be 0 to make the domain of the gcd operator all 
pairs of natural numbers. 

Let $m, n, k \in \mathbb{N}$, not all of which are 0, and suppose $k = $ gcd$(m,n)$. 
$$k \mid m \land k \mid n \land (\forall e \in \mathbb{N}, e \mid m \land e 
\mid n \Rightarrow e \leq k)$$

\ex Prove that for all natural numbers $p, q$, if $p$ and $q$ are distinct 
primes, then gcd$(p, q) = 1$.

\textit{Translation.} We can translate the statement without unpacking 
definitions.\sidenote{Unpacking definitions here would obscure the meaning 
of the statement and overcomplicating it --- we will use the definitions 
in the proof, so it is meaningless to include it here.}
$$\forall p, q \in \mathbb{N}, (Prime(p) \land Prime(q) \land p \neq q) 
\Rightarrow \text{gcd} (p, q ) = 1$$

\textit{Discussion.} We know that primes only have two divisors, 1 and itself. 
We just need to make sure that neither $p $ or $q$ divides the other, since 
that would make the gcd larger than 1. 

\textit{Proof.} Let $p, q \in \mathbb{N}$. Assume that both $p$ and $q$ are 
prime, and that they are unique. We want to prove that gcd$(p, q) = 1$. 

By the definition of primality, $p$ and $q$ cannot be 1, and the only 
positive divisors of $p$ and $q$ are 1 and themselves. Since $p \neq q$ 
and $p \neq 1$, we know that $p \nmid q$. 

Thus, gcd$(p, q) = 1$ since the only positive divisors of $p$ and $q$ is 1. 

\tr Let $a, b \in \mathbb{N}$, assume at least one of them is non-zero. 
Then gcd$(a, b)$ is the smallest positive integer such that $\exists p, q 
\in \mathbb{Z}$ with gcd$(a, b) = ap + bq$. In other words, the gcd is the 
smallest natural number that can be written as the sum of (positive or negative) 
multiple of the two numbers.

We will now use this theorem to introduce a new proof technique --- \textbf{using 
an external statement} as a step in a proof. We do not need to know how 
the external theorem is implemented, only to know what it means. 

\ex For all $a, b \in \mathbb{N}$, every integer that divides both $a, b$
also divides gcd$(a, b)$. 

\textit{Translation.} $\forall a, b \in \mathbb{N}, \forall d \in \mathbb{Z}, 
(d \mid a \land d \mid b) \Rightarrow d \mid \text{gcd}(a, b)$

\textit{Discussion.} All we know from gcd is that $d \leq $gcd$(a, b)$, 
and that does not imply $d \mid$ gcd$(a, b)$. We can use the \textbf{GCD 
Characterization Theorem} to write gcd$(a, b)$ as $ap + bq$.

\textit{Proof.} Let $a, b \in \mathbb{N}$ and $d \in \mathbb{Z}$. Assume 
that $d \mid a$ and $d \mid b$. We want to prove that $d \mid$ gcd$(a, b)$. 

By the GCD Characterization Theorem, $\exists p, q \in \mathbb{Z}$ such that 
gcd$(a, b) = ap + bq$.\sidenote{Using the external theorem} By divisibility 
of linear combinations, since $d \mid a$ and $d \mid b$, 
we know $d \mid ap + bq$. Since gcd$(a, b) = ap + bq$, we can conclude that 
\linebreak $d \mid $ gcd$(a, b)$.

\newpage
\section{Modular arithmetic}
When dealing with relationships between numbers, the divisibility predicate
is often too constrained by the binary nature of its output. Instead, 
we often care more about the \textit{remainder} when we divide a number 
by another. 

\df Let $a, b, n \in \mathbb{Z}$, with $n \neq 0$. We say that $a$ is \linebreak
\textbf{congruent to} $b$ \textbf{modulo} $n$ iff $n \mid a - b$. We can 
write $a \equiv b$ \linebreak (mod $n$).\sidenote{The notation $a \equiv b$ is 
not entirely the same as the \% operator used in programming: both $a$ and 
$b$ could be much larger than $n$, or even negative.} 

In other words, $a$ and $b$ have the same remainder when divided by $n$. 

$a = b$ \hfill Numeric values are literally equal. \\
$a \equiv b$ (mod $n$) \hfill Remainders when divided by $n$ are equal. 

We will now look at how addition, subtraction, and multiplication behave 
under modular arithmetic. 

\ex Prove that for all $a, b, c, d, n \in \mathbb{Z}$, with $n \neq 0$, if 
$a \equiv c$ (mod $n$) and $b \equiv d$ (mod $n$):
\begin{enumerate}
    \item $a + b \equiv c + d$ (mod $n$)
    \item $a - b \equiv c - d$ (mod $n$)
    \item $ab \equiv cd$ (mod $n$)
\end{enumerate}
\textit{Translation.} We will only show how to unpack (2), the rest are similar:
$$\forall a, b, c, d, n \in \mathbb{Z}, (n \neq 0 \land n \mid (a-c)\land 
n \mid (b-d) ) \Rightarrow n \mid ((a - b) - (c - d))$$
\textit{Proof.} Let $a, b, c, d, n \in \mathbb{Z}$. Assume $n \neq 0$ and 
$n \mid (a -c)$ and $n \mid (c-d)$. We want to prove that $n \mid ((a-c) - 
(b - d))$. By the divisibility of linear combinations, since $n \mid (a-c)$ 
and $n \mid (b-d)$, it will also divide their difference. 

\ex Let $a, b, p \in \mathbb{Z}$. If $p$ is a prime number and $a$ is not 
divisible by $p$, then there exists $k \in \mathbb{Z}$ such that $ak \equiv 
b$ (mod $p$).

\textit{Translation.} $\forall a, b, p, \in \mathbb{Z},(Prime(p) \land 
p \nmid a )\Rightarrow (\exists k \in \mathbb{Z}, ak \equiv b \;(\text{mod } p))$

\textit{Discussion.} We know we will have to use the definition of congruence 
like "there exists $k \in \mathbb{Z}$ such that ..." We will also use the 
GCD Characterization Theorem to write gcd as a sum of multiples. We can 
assume $p$ is prime, $p \nmid a$, and gcd of two numbers can be written 
as the sum of multiples of the numbers.

We want to prove that:
\begin{itemize}
    \item $\exists k \in \mathbb{Z}, ak \equiv 1$ (mod $p$), which is equivalent to 
    \item $\exists k \in \mathbb{Z}, p \mid (ak - 1)$ using the definition of 
        mod. that is equal to 
    \item $\exists k, d \in \mathbb{Z}, dp = ak - 1$ 
    \item $\exists k, d \in \mathbb{Z}, 1 = ak - dp$ which is a sum of multiples
\end{itemize}

\textit{Proof.} Let $a, b, p \in \mathbb{N}$. Assume $p$ is prime and $p$ 
does not divide $a$. We want to prove that $ak \equiv b$ (mod $p$). We need 
to prove two subclaims in order to complete this proof.\sidenote{We can think 
of these as \textbf{helper functions} in terms of progrmaming, to break up 
the proof in smaller steps.}

\textit{Claim 1.} gcd$(a, p) = 1$.

\textit{Proof.} By the defintion of prime, we know that the greatest divisor 
that is not itself is 1. Since we assumed that $p \nmid a$, we know that 
1 is the only positve common divisor. 

\textit{Claim 2.} $\exists k \in \mathbb{Z}, ak \equiv 1$ (mod $p$).

\textit{Proof.} By the previous claim, we know gcd$(a, p) =1$. By Theorem 2.1 
(QRT), we know that there exists $r, s \in \mathbb{Z}$ such that $ar + ps = 1.$

Let $k = r$, then we can rearrange 
\begin{align*}
    ak + ps &= 1 \\
    ak - 1 &= p(-s) \\
    p &\mid (ak -1) \\
    ak &\equiv 1 \:(\text{mod }p) 
\end{align*}

Finally, we can combine these claims to prove there exists a $k' \in \mathbb{Z}$ 
such that $ak' = b$ (mod $p$).

Let $k' = kb$. Then we have 
\begin{align*}
    ak \equiv 1 \:(\text{mod } p) \\
    akb \equiv b \:(\text{mod } p) \\
    ak' \equiv b \:(\text{mod } p) 
\end{align*}

\section{Proof by contradiction}
\df A \textbf{proof of contradiction} states that given a statement $P$ to 
prove, rather than proving it directly we assume that its \textbf{negation
$\neg P$} is True. We use this assumption to prove a statement $Q$ and its 
negation $\neg Q$. We call $Q$ and $\neg Q$ the \textbf{contradiction} that 
arises from the assumption that $P$ is False.\sidenote{In other words, "if 
$P$ is False, then $Q$ and its negation $\neg Q$ must be True."}

\tr There are infinitely many primes.

\textit{Proof.} Assume that this statement is False. Let $k \in \mathbb{N}$ be
the number of primes, and let $p_1, p_2, \ldots, p_k$ be the prime numbers. 

Let $Q : \forall n \in \mathbb{N}, Prime(n) \Leftrightarrow (n \in p_1, p_2,
\ldots p_k)$. Q is True because our asumption that there is a finite number of 
primes. 

Now we will show that $Q$ is False. Define the number 
$$P = 1 + \prod_{i = 1}^{k} p_i = 1 + p_1 \times p_2 \times \ldots \times p_k $$
There must be some prime $p$ that divides $P$ because $P > 1$. But $p \notin 
{p_1, \ldots, p_k}$ since otherwise $p$ would divide $P - p_1 \times \ldots 
\times p_k = 1$, and no prime can divide 1. So $p$ is a prime that is not 
one of ${p_1 \times \ldots \times p_k}$, so $Q$ is False, thus there is a 
contradiction. 


\end{document}
